%%%%%%%%%%%%%%%%%%%%%%%%%%%%%%%%%%%%%%%%%%%%%%%%%%%%%%%%%%%%%%%%%%%%%%%%%%
\chapter{Nástroje pro manipulaci s bajtkódem}

% Kapitola o existujících nástrojích pro manipulaci s bajtkódem.
% Vybrala jsem tři populární nástroje, které se navzájem velmi liší.
% BCEL a ACM jsou nízkoúrovňové nástroje a pro manipulaci s kódem je třeba znalost bajtkódu.
% Javassist umožňuje pracovat s kódem na vyšší úrovni.

% Uveden0 knihovny jsou implementované v Jazyce Java

% zm9nit dal39 n8stroje

\section{BCEL}

% Z jakých nejdůležitějších tříd se knihovna skládá?
% Jakým způsobem lze manipulovat s bajtkódem?
% Jaké další nástroje BCEL nabízí?


Byte Code Engineering Library (BCEL) je jedna z knihoven, které umožňují pracovat s obsahem \texttt{class} souborů. Knihovna spadá do projektu Apache Commons a je poskytovaná pod licencí Apache License 2.0. Poslední verze BCEL 5.2 nepodporuje Javu 8, ale z repozitáře je dostupná verze 6.0, kde je podpora implementovaná. Vývoj knihovny však v posledních letech není příliš aktivní. 

Programové rozhraní knihovny je dostupné v balíčku \texttt{org.apache.bcel}. Knihovna obsahuje třídy pro statický popis \texttt{class} souborů, třídy pro dynamické úpravy a vytváření bajtkódu a třídy s různými nástroji. Syntaktickou analýzu \texttt{class} souboru a vytvoření reprezentace jeho obsahu v podobě instance třídy \texttt{JavaClass} umožňuje třída \texttt{ClassParser} z balíčku \texttt{org.apache.bcel.classfile}. Součastí balíčku jsou současně všechny třídy podílející se na popisu obsahu souboru. Pro každou položku souboru je tedy vytvořen nový objekt. Takový přístup může být velmi neefektivní, pokud je třeba zpracovat velké množství souborů (?). Na druhou stranu třída \texttt{JavaClass} velmi přesně kopíruje formát \texttt{class} souboru tak, jak byl popsáný v kapitole (?), včetně tabulky konstant.
Pro dynamické vytváření a úpravu bajtkódu je třeba vyšší míra abstrakce. Tu poskytují třídy z balíčku \texttt{org.apache.bcel.generic}. Pomocí těchto tříd je třeba sestavit celý obsah \texttt{class} souboru včetně tabulky konstant. Korektnost výsledného bajtkódu lze zkontrolovat třídou \texttt{Verifier} z balíčku \texttt{org.apache.bcel.verifier}.

%ClassGen, MethodGen, InstructionHandler, InstructionList, 
% zmínit visitora, verified, BCELLifier, instruction finder

Knihovna BCEL poskytuje pro bajtkód velmi nízkou úroveň abstrakce. Je třeba být seznámen s formátem \texttt{class} souborů a pracovat s tabulkou konstant. Bajtkód je navíc reprezentovaný velkým množstvím objektů a neexistuje efektivní způsob, jak zpracovat jen ty informace, které jsou v dané aplikaci potřeba. Vhodnou alternativou je proto knihovna ASM.


% příklad načtení kódu a výpis 
% příklad vytvoření kódu ?




% FIndBugs

\section{ACM}

% Z jakých nejdůležitějších tříd se knihovna skládá?
% Jakým způsobem lze manipulovat s bajtkódem?
% Jaké další nástroje ACM nabízí?
% zm9nit licenci

\section{Javassist}

% Z jakých nejdůležitějších tříd se knihovna skládá?
% Jakým způsobem lze manipulovat s bajtkódem?
% Jaké další nástroje Javassist nabízí?
% zm9nit licenci


%=========================================================================
