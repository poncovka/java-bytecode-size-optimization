%%%%%%%%%%%%%%%%%%%%%%%%%%%%%%%%%%%%%%%%%%%%%%%%%%%%%%%%%%%%%%%%%%%%%%%%%%
\chapter{Nástroje pro manipulaci s bajtkódem}

% Kapitola o existujících nástrojích pro manipulaci s bajtkódem.
% Vybrala jsem tři populární nástroje, které se navzájem velmi liší.
% BCEL a ACM jsou nízkoúrovňové nástroje a pro manipulaci s kódem je třeba znalost bajtkódu.
% Javassist umožňuje pracovat s kódem na vyšší úrovni.

\section{BCEL}

% Z jakých nejdůležitějších tříd se knihovna skládá?
% Jakým způsobem lze manipulovat s bajtkódem?
% Jaké další nástroje BCEL nabízí?

\section{ACM}

% Z jakých nejdůležitějších tříd se knihovna skládá?
% Jakým způsobem lze manipulovat s bajtkódem?
% Jaké další nástroje ACM nabízí?


\section{Javassist}

% Z jakých nejdůležitějších tříd se knihovna skládá?
% Jakým způsobem lze manipulovat s bajtkódem?
% Jaké další nástroje Javassist nabízí?

\section{Srovnání nástrojů}

% Srovnání výhod a nevýhod jednotlivých nástrojů. Za jakých podmínek je daný nástroj vhodnější?
% Případně zmínka o dalších nástrojích.

%=========================================================================
