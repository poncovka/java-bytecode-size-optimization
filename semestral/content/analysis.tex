%%%%%%%%%%%%%%%%%%%%%%%%%%%%%%%%%%%%%%%%%%%%%%%%%%%%%%%%%%%%%%%%%%%%%%%%%%
\chapter{Analýza bajtkódu}

% Popis způsobu analýzy rozsáhlého vzorku testovacích dat, prezentace výsledků a zhodnocení.

\section{Nástroje}

% Zde bude uveden popis nástrojů, které jsem použila k analýze bajtkódu. 
% Jedná se o nástroje, které jsem sama implementovala, ale popis implementace v této práci příliš rozvádět nechci.

\subsubsection{Načtení testovacích dat}

% Jakým způsobem načítám data z class souborů?

\subsubsection{Zpracování bajtkódu}

% Jakým způsobem a pomocí kterého nástroje (ACM) zpracovávám data?

\subsubsection{Nalezení typických sekvencí instrukcí}

% Jak vypadá graf podřetězců? Proč je pro analýzu vhodnější než hash tabulka s podřetězci jako klíči a četnostmi výskytu jako hodnotami?
% Jak lze pomocí grafu podřetězců získat typické sekvence instrukcí? Jak lze graf zobrazit?

\section{Testovací data}

% Jaká data jsem použila k analýze? Odkud jsem je získala? Jakých kritérií jsem se držela?

\section{Výsledky analýzy}

% Zde budou uvedeny tabulky a grafy s výsledky analýzy.

\section{Vyhodnocení}

% Shrnutí výsledků a komentář. Jak mohu výsledky analýzy použít pro další postup práce?

%=========================================================================
