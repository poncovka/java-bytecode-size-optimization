%%%%%%%%%%%%%%%%%%%%%%%%%%%%%%%%%%%%%%%%%%%%%%%%%%%%%%%%%%%%%%%%%%%%%%%%%%
\chapter{Java Virtual Machine}

% Základní charakteristika JVM.
% Tato kapitola slouží pouze k doplnění kontextu.

\section{Architektura Javy}

% Krátce historie, popis jednotlivých částí Javy: jazyk Java, instrukční soubor class, překladač, JVM, Java API).
% Uvedena verze Javy, ze které jsem vycházela. Kapitola slouží pouze k doplnění kontextu.

\section{Datové typy}

% Datové typy podporované JVM a jejich reprezentace.

\section{Paměťové oblasti}

% Popis paměťových oblastí JVM (zásobník, halda, constant pool, rámce, lokální proměnné, zásobník operandů)
% Co JVM dělá při volání metody a návratu?
% Co JVM dělá, když je vyvolaná výjimka?

\section{Verifikace instrukčního souboru}

% Krátký popis omezení, která jsou kladená na instrukční soubor, a verifikace, kterou JVM provádí.

\section{Načtení, sestavení a inicializace tříd a rozhraní}

% Jak probíhá dynamické vytváření tříd a rozhraní?

%=========================================================================
