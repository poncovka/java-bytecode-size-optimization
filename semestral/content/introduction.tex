%%%%%%%%%%%%%%%%%%%%%%%%%%%%%%%%%%%%%%%%%%%%%%%%%%%%%%%%%%%%%%%%%%%%%%%%%%
\chapter{Úvod}

% Úvod do tématu. Co je cílem této semestrální a následující diplomové práce? Stručný popis jednotlivých kapitol.

V této práci se zabývám bajtkódem Javy z hlediska optimalizace výsledné velikosti. V kapitole () popisuji formát, v jakém je bajtkód uložen v instrukčních souborech a v kapitole () způsob, jakým je bajtkód interpretován virtuálním strojem Java Virtual Machine. Dále v kapitole () uvádím stručný popis nástrojů pro manipulaci s bajtkódem a shrnuji jejich výhody a nevýhody. Konkrétně zmiňuji nástroje BCEL, ACM a Javassist. Kapitola ()  je věnovaná návrhu a implementaci nástroje pro analýzu bajtkódu. Pomocí tohoto nástroje jsem získala data, která jsem dále analyzovala v kapitole (). Cílem této práce bylo seznámit se s bajtkódem a diagnostikovat místa, která by bylo vhodné optimalizovat.

%=========================================================================

