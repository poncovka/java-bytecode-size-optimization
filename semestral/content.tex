%=========================================================================

%%%%%%%%%%%%%%%%%%%%%%%%%%%%%%%%%%%%%%%%%%%%%%%%%%%%%%%%%%%%%%%%%%%%%%%%%%%
\chapter{Úvod}\label{Introduction}

% Úvod do tématu. 
TODO

%V~této práci se zabývám bajtkódem Javy z~hlediska optimalizace jeho velikosti. V~kapitole~\ref{JVM} popisuji virtuální stroj Java Virtual Machine a způsob, jakým je bajtkód interpretován. Kapitola \ref{Format} je věnovaná formátu, v~jakém je bajtkód uložen v~instrukčních souborech. Dále uvádím stručný popis nástrojů pro manipulaci s~bajtkódem a shrnuji jejich výhody a nevýhody v~kapitole \ref{Tools}. Konkrétně zmiňuji nástroje BCEL, ACM a Javassist. Kapitola \ref{Tool} je věnovaná návrhu a implementaci nástroje pro analýzu bajtkódu. Pomocí tohoto nástroje jsem získala data, která jsem zpracovala a vyhodnotila v~kapitole \ref{Analysis}. Cílem této práce je seznámení se s~bajtkódem a diagnostika míst vhodných k~optimalizaci z~hlediska velikosti.

%=========================================================================


%%%%%%%%%%%%%%%%%%%%%%%%%%%%%%%%%%%%%%%%%%%%%%%%%%%%%%%%%%%%%%%%%%%%%%%%%%%
\chapter{Závěr}

% Shrnutí toho, co jsem udělala, jakých výsledků jsem dosáhla a jak budu pokračovat.


%=========================================================================



\chapter{Úvod}

Úvod do tématu. Co je cílem této semestrální a následující diplomové práce? Stručný popis jednotlivých kapitol.

\chapter{Architektura Javy}

Krátce historie, popis jednotlivých částí Javy: jazyk Java, instrukční soubor class, překladač, JVM, Java API).
Uvedena verze Javy, ze které jsem vycházela.

\chapter{Java Virtual Machine}

Základní charakteristika JVM.

\section{Datové typy}

Datové typy podporované JVM a jejich reprezentace.

\section{Paměťové oblasti}

Popis paměťových oblastí JVM (zásobník, halda, constant pool, rámce, lokální proměnné, zásobník operandů)

\section{Volání metody}

Co JVM dělá při volání metody a návratu?

\section{Vyvolání výjimky}

Co JVM dělá, když je vyvolaná výjimka?

\chapter{Formát instrukčního souboru}

Zde je popsaná struktura class souboru.

\section{Základní struktura}

Jaká je základní struktura class souboru?

\section{Konstanty}

Popis struktury constant\_pool. Jakým způsobem jsou ukládané konstanty? Jak jsou reprezentované různé typy?

\section{Členské proměnné}

Popis struktury field\_info. Jakým způsobem jsou uložené členské proměnné?

\section{Metody}

Popis struktury method\_info. Jakým způsobem jsou uložené metody? Kde jsou uloženy instrukce?

\section{Instrukce}

Přehled instrukcí a krátké příklady bajkódu.

\section{Atributy}

Základní přehled atributů, jen ty podstatné, neboť jich je mnoho.


\chapter{Nástroje pro manipulaci s bajtkódem}

Kapitola o existujících nástrojích pro manipulaci s bajtkódem.
Vybrala jsem tři populární nástroje, které se navzájem velmi liší.
BCEL a ACM jsou nízkoúrovňové nástroje a pro manipulaci s kódem je třeba znalost bajkódu.
Javassist umožňuje pracovat s kódem na vyšší úrovni.

\section{BCEL}

Z jakých nejdůležitějších tříd se knihovna skládá?
Jakým způsobem lze manipulovat s bajtkódem?
Jaké další nástroje BCEL nabízí?

\section{ACM}

Z jakých nejdůležitějších tříd se knihovna skládá?
Jakým způsobem lze manipulovat s bajtkódem?
Jaké další nástroje ACM nabízí?


\section{Javassist}

Z jakých nejdůležitějších tříd se knihovna skládá?
Jakým způsobem lze manipulovat s bajtkódem?
Jaké další nástroje Javassist nabízí?

\section{Srovnání nástrojů}

Srovnání výhod a nevýhod jednotlivých nástrojů. Za jakých podmínek je daný nástroj vhodnější?
Případně zmínka o dalších nástrojích.


\chapter{Analýza bajtkódu}

Popis způsobu analýzy rozsáhlého vzorku testovacích dat, prezentace výsledků a zhodnocení.

\section{Nástroje}

Zde bude uveden popis nástrojů, které jsem použila k analýze bajkódu. 
Jedná se o nástroje, které jsem sama implementovala, ale popis implementace v této práci příliš rozvádět nechci.

\subsubsection{Načtení testovacích dat}

Jakým způsobem načítám data z class souborů?

\subsubsection{Zpracování bajtkódu}

Jakým způsobem a pomocí kterého nástroje (ACM) zpracovávám data?

\subsubsection{Nalezení typických sekvencí instrukcí}

Jak vypadá graf podřetězců? Proč je pro analýzu vhodnější než hash tabulka s podřetězci jako klíči a četnostmi výskytu jako hodnotami?
Jak lze pomocí grafu podřetězců získat typické sekvence instrukcí? Jak lze graf zobrazit?

\section{Testovací data}

Jaká data jsem použila k analýze? Odkud jsem je získala? Jakých kritérií jsem se držela?

\section{Výsledky analýzy}

Zde budou uvedeny tabulky a grafy s výsledky analýzy.

\section{Vyhodnocení}

Shrnutí výsledků a komentář. Jak mohu výsledky analýzy použít pro další postup práce?


\chapter{Závěr}

Shrnutí toho, co jsem udělala, jakých výsledků jsem dosáhla a jak budu pokračovat.

%=========================================================================
