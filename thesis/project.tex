%============================================================================
% tento soubor pouzijte jako zaklad
% (c) 2008 Michal Bidlo
% E-mail: bidlom AT fit vutbr cz
%============================================================================
% kodovaní: UTF-8 (zmena prikazem iconv, recode nebo cstocs)
%----------------------------------------------------------------------------
% zpracování: make, make pdf, make desky, make clean
%============================================================================
% Šablonu upravil: Ing. Jaroslav Dytrych, idytrych@fit.vutbr.cz
%============================================================================

%nezpracovava se pomoci pdflatex
%\let\pdfoutput\undefined

\documentclass[zadani,print]{fitthesis} % bez zadání - pro začátek práce, aby nebyl problém s překladem
%\documentclass[zadani]{fitthesis} % odevzdani do wisu - odkazy jsou barevné
%\documentclass[zadani,print]{fitthesis} % pro tisk - odkazy jsou černé
%\documentclass[english,print]{fitthesis} % pro tisk - odkazy jsou černé
% * Je-li prace psana v anglickem jazyce, je zapotrebi u tridy pouzit 
%   parametr english nasledovne:
%      \documentclass[english]{fitthesis}

\usepackage[czech,english]{babel}
\usepackage[utf8]{inputenc} %kodovani
\usepackage[T1, IL2]{fontenc}
\usepackage{cmap}
\usepackage{url}
\DeclareUrlCommand\url{\def\UrlLeft{<}\def\UrlRight{>} \urlstyle{tt}}

%zde muzeme vlozit vlastni balicky

%--------------------------
\usepackage{amsthm}
\usepackage{amsmath}
\usepackage{mdwlist}
\usepackage[bf]{caption}
\usepackage{tikz}
\usepackage{tikz-qtree}
\usetikzlibrary{trees}
\usepackage{epstopdf}
\usepackage{makecell}

%\theoremstyle{definition}
%\newtheorem{Def}{Definice}[section]
%\newtheorem{Alg}{Algoritmus}
%\newtheorem{Example}{Příklad}[section]
%\newtheorem{Veta}{Věta}

%\newcommand{\nt}[1]{$<$\textit{#1}$>$}
\newcommand{\N}[1]{\textit{#1}}
\newcommand{\T}[1]{\texttt{#1}}

\newenvironment{texamples} {
  \newcommand{\code}[1]{\texttt{\makecell*[tp{3.5cm}]{##1}}}
  \newcommand{\desc}[1]{\makecell*[tp{9cm}]{##1}}
  \begin{center}
  \begin{tabular}{ r  l  l } \hline
  \textbf{Výskyt} & \textbf{Sekvence} & \textbf{Popis sekvence} \\\hline
} {
  \hline
  \end{tabular}
  \end{center}
}

\newenvironment{tpatterns} {
  \newcommand{\code}[1]{\texttt{\makecell*[tp{3.5cm}]{##1}}}
  \newcommand{\desc}[1]{\makecell*[tp{6.7cm}]{##1}}
  \begin{center}
  \begin{tabular}{l  l  l } \hline
  \textbf{Vzor} & \textbf{Náhrada} & \textbf{Popis optimalizace} \\\hline
} {
  \hline
  \end{tabular}
  \end{center}
}

%--------------------------

\usepackage{listings}
\usepackage[toc,page,header]{appendix}
\RequirePackage{titletoc}
\ifczech
  \usepackage{ae}
\fi

%---rm---------------
%\renewcommand{\rmdefault}{lmr}%zavede Latin Modern Roman jako rm
%---sf---------------
%\renewcommand{\sfdefault}{qhv}%zavede TeX Gyre Heros jako sf
%---tt------------
%\renewcommand{\ttdefault}{lmtt}% zavede Latin Modern tt jako tt

% vypne funkci nové šablony, která automaticky nahrazuje uvozovky,
% aby nebyly prováděny nevhodné náhrady v popisech API apod.
\csdoublequotesoff

% =======================================================================
% balíček "hyperref" vytváří klikací odkazy v pdf, pokud tedy použijeme pdflatex
% problém je, že balíček hyperref musí být uveden jako poslední, takže nemůže
% být v šabloně
\ifWis
\ifx\pdfoutput\undefined % nejedeme pod pdflatexem
\else
  \usepackage{color}
  \usepackage[unicode,colorlinks,hyperindex,plainpages=false,pdftex]{hyperref}
  \definecolor{links}{rgb}{0.4,0.5,0}
  \definecolor{anchors}{rgb}{1,0,0}
  \def\AnchorColor{anchors}
  \def\LinkColor{links}
  \def\pdfBorderAttrs{/Border [0 0 0] }  % bez okrajů kolem odkazů
  \pdfcompresslevel=9
\fi
\else % pro tisk budou odkazy, na které se dá klikat, černé
\ifx\pdfoutput\undefined % nejedeme pod pdflatexem
\else
  \usepackage{color}
  \usepackage[unicode,colorlinks,hyperindex,plainpages=false,pdftex,urlcolor=black,linkcolor=black,citecolor=black]{hyperref}
  \definecolor{links}{rgb}{0,0,0}
  \definecolor{anchors}{rgb}{0,0,0}
  \def\AnchorColor{anchors}
  \def\LinkColor{links}
  \def\pdfBorderAttrs{/Border [0 0 0] } % bez okrajů kolem odkazů
  \pdfcompresslevel=9
\fi
\fi

%Informace o praci/projektu
%---------------------------------------------------------------------------
\projectinfo{
  %Prace
  project=DP,            %typ prace BP/SP/DP/DR
  year=2016,             %rok
  date=\today,           %datum odevzdani
  %Nazev prace
  title.cs={Optimalizace velikosti bajtkódu Javy},  %nazev prace v cestine
  title.en={Java Bytecode Size Optimization}, %nazev prace v anglictine
  %Autor
  author={Vendula Poncová},   %jmeno prijmeni autora
  author.title.p=Bc., %titul pred jmenem (nepovinne)
  %author.title.a=PhD, %titul za jmenem (nepovinne)
  %Ustav
  department=UITS, % doplnte prislusnou zkratku: UPSY/UIFS/UITS/UPGM
  %Skolitel
  supervisor=Radek Kočí, %jmeno prijmeni skolitele
  supervisor.title.p=Ing.,   %titul pred jmenem (nepovinne)
  supervisor.title.a={Ph.D.},    %titul za jmenem (nepovinne)
  %Klicova slova, abstrakty, prohlaseni a podekovani je mozne definovat 
  %bud pomoci nasledujicich parametru nebo pomoci vyhrazenych maker (viz dale)
  %===========================================================================
  %Klicova slova
  %keywords.cs={Klíčová slova v českém jazyce.}, %klicova slova v ceskem jazyce
  %keywords.en={Klíčová slova v anglickém jazyce.}, %klicova slova v anglickem jazyce
  %Abstract
  %abstract.cs={Výtah (abstrakt) práce v českém jazyce.}, % abstrakt v ceskem jazyce
  %abstract.en={Výtah (abstrakt) práce v anglickém jazyce.}, % abstrakt v anglickem jazyce
  %Prohlaseni
  %declaration={Prohlašuji, že jsem tuto bakalářskou práci vypracoval samostatně pod vedením pana ...},
  %Podekovani (nepovinne)
  %acknowledgment={Zde je možné uvést poděkování vedoucímu práce a těm, kteří poskytli odbornou pomoc.} % nepovinne
}

%Abstrakt (cesky, anglicky)
\abstract[cs]{Tato práce se zabývá bajtkódem jazyka Java z~hlediska jeho velikosti. Popisuje virtuální stroj Javy a formát jeho instrukčního souboru a uvádí přehled některých knihoven pro manipulaci s~bajtkódem. S~pomocí těchto knihoven byla provedena analýza vybraného vzorku dat a nalezeny sekvence instrukcí, které by bylo možné optimalizovat. Na základě výsledků analýzy byly navrhnuty a implementovány metody pro optimalizaci velikosti bajtkódu. Velikost bajtkódu zkoumaného vzorku dat se po aplikaci metod snížila o~zhruba 25\%.}
\abstract[en]{This paper deals with the Java bytecode in terms of its size. It describes the Java Virtual Machine and the Java class file format. It also presents some tools for the bytecode manipulation. Using these tools, I~have analyzed selected data and found sequences of instructions, that could be optimized. Based on the results of the analysis, I~have designed and implemented methods for bytecode size optimization. The bytecode size of the selected data was reduced by roughly 25\%.}

%Klicova slova (cesky, anglicky)
\keywords[cs]{Java, JVM, bajtkód, ASM, BCEL, Javassist, optimalizace velikosti, peephole optimalizace}
\keywords[en]{Java, JVM, bytecode, ASM, BCEL, Javassist, size optimization, peephole optimization}

%Prohlaseni
\declaration{Prohlašuji, že jsem tuto semestrální práci vypracovala samostatně pod vedením pana Ing. Radka Kočího, Ph.D. a pana Ing. Pavla Tišnovského, Ph.D. Uvedla jsem všechny literární prameny a publikace, ze kterých jsem čerpala.}

%Podekovani
\acknowledgment{Děkuji panu Ing. Radkovi Kočímu za odborné vedení a pomoc při zpracování této práce. Mé poděkování patří též panu Ing. Pavlu Tišnovskému za cenné rady a připomínky.}

\begin{document}
  % Vysazeni titulnich stran
  % ----------------------------------------------
  \maketitle
  % Obsah
  % ----------------------------------------------
  \tableofcontents
  
  % Seznam obrazku a tabulek (pokud prace obsahuje velke mnozstvi obrazku, tak se to hodi)
\ifczech
  \renewcommand\listfigurename{Seznam obrázků}
\fi
  % \listoffigures
\ifczech
  \renewcommand\listtablename{Seznam tabulek}
\fi
  % \listoftables 

  % Text prace
  % ----------------------------------------------
 %=========================================================================

%%%%%%%%%%%%%%%%%%%%%%%%%%%%%%%%%%%%%%%%%%%%%%%%%%%%%%%%%%%%%%%%%%%%%%%%%%
\chapter{Úvod}\label{Introduction}

% Úvod do tématu. 
TODO

%V~této práci se zabývám bajtkódem Javy z~hlediska optimalizace jeho velikosti. V~kapitole~\ref{JVM} popisuji virtuální stroj Java Virtual Machine a způsob, jakým je bajtkód interpretován. Kapitola \ref{Format} je věnovaná formátu, v~jakém je bajtkód uložen v~instrukčních souborech. Dále uvádím stručný popis nástrojů pro manipulaci s~bajtkódem a shrnuji jejich výhody a nevýhody v~kapitole \ref{Tools}. Konkrétně zmiňuji nástroje BCEL, ACM a Javassist. Kapitola \ref{Tool} je věnovaná návrhu a implementaci nástroje pro analýzu bajtkódu. Pomocí tohoto nástroje jsem získala data, která jsem zpracovala a vyhodnotila v~kapitole \ref{Analysis}. Cílem této práce je seznámení se s~bajtkódem a diagnostika míst vhodných k~optimalizaci z~hlediska velikosti.

%=========================================================================


%%%%%%%%%%%%%%%%%%%%%%%%%%%%%%%%%%%%%%%%%%%%%%%%%%%%%%%%%%%%%%%%%%%%%%%%%%
\chapter{Java Virtual Machine}

% Základní charakteristika JVM.
% Tato kapitola slouží pouze k doplnění kontextu.

\section{Datové typy}

% Datové typy podporované JVM a jejich reprezentace.

\section{Paměťové oblasti}

% Popis paměťových oblastí JVM (zásobník, halda, constant pool, rámce, lokální proměnné, zásobník operandů)
% Co JVM dělá při volání metody a návratu?
% Co JVM dělá, když je vyvolaná výjimka?

\section{Verifikace instrukčního souboru}

% Krátký popis omezení, která jsou kladená na instrukční soubor, a verifikace, kterou JVM provádí.

\section{Načtení, sestavení a inicializace tříd a rozhraní}

% Jak probíhá dynamické vytváření tříd a rozhraní?

%=========================================================================

%%%%%%%%%%%%%%%%%%%%%%%%%%%%%%%%%%%%%%%%%%%%%%%%%%%%%%%%%%%%%%%%%%%%%%%%%%
\chapter{Formát instrukčního souboru}

% TODO
% * dopsat info u položek
% * definovat byte
% * nastudovat instrukce
% * ověřit velikosti "sdílených" nonterminálů !!!

Při kompilaci \texttt{java} souboru překladač pro každou definovanou třídu a rozhraní vytvoří jeden soubor s příponou \texttt{.class} (dále \texttt{class} soubor). Tento soubor obsahuje binární reprezentaci kompilovaného mezikódu, který lze interpretovat prostřednictvím JVM. V této kapitole popisuji formát \texttt{class} souboru dle specifikace ve verzi Java SE 8 Edition(X).

Pro popis formátu jsem zvolila rozšířenou Backus-Naurovu formu (?), která umožňuje zapsat syntaxi formálního jazyka pomocí pravidel a terminálních a nonterminálních symbolů. Nonterminální symboly jsou definovány pomocí definujícího symbolu \texttt{:=}, symbolu pro konkatenaci \texttt{,}, symbolu pro alternaci \texttt{$\mathtt{|}$}, symbolů pro nula a více opakování \texttt{\{\}}, ukončujícího symbolu \texttt{;} a pomocí graficky odlišených \T{terminálních} a \N{nonterminálních} symbolů.

\section{Základní struktura}

% Jaká je základní struktura class souboru?

Položky \texttt{class} souboru tvoří posloupnost osmibitových bajtů. Základní stavební jednotkou je tedy bajt, který je v pravidlech reprezentovaný symbolem \N{B}. Symbol $\langle n \rangle$\N{B}, kde $\langle n \rangle \in \{2,3,\dots\}$, reprezentuje $n$ bajtů. Terminály jsou hexadecimální reprezentací posloupnosti bajtů s prefixem \texttt{0x}. 

Základní struktura souboru je popsaná pravidlem pro symbol \N{classfile}. Soubor obsahuje informace o jeho typu a verzi (\N{version}), disponuje tabulkou všech konstant, které se v souboru vyskytují (\N{constants}), nese informace o třídě, kterou reprezentuje, případně rozhraní (\N{class}), obsahuje seznam rozhraní, které reprezentovaná třída implementuje, případně rozhraní rozšiřuje  (\N{interface\_list}), seznam členských proměnných třídy (\N{field\_list}), seznam metod (\N{method\_list}) a seznam atributů (\N{attribute\_list}). 

\begin{figure}[h!]
  \begin{tabular}{r c l}
  \N{classfile} &:=& \N{version}, \N{constants}, \N{class}, \N{interface\_list}, \N{field\_list}, \N{method\_list}, \N{attribute\_list};
  \end{tabular}
\end{figure}

Typ souboru je definován prvními čtyřmi bajty, které jsou popsané symbolem \N{magic\_number}. Verze souboru je tvořena hodnotou $M$ symbolu \N{major\_version} a hodnotou $m$ symbolu \N{minor\_version} jako $M.m$.

\begin{figure} [h!]
  \begin{tabular}{r c l}
  \N{version} &:=& \N{magic\_number}, \N{minor\_version}, \N{major\_version};\\
  \N{magic\_number} &:=& \T{0xCAFEBABE};\\
  \N{minor\_version} &:=& \N{2B};\\
  \N{major\_version} &:=& \N{2B};\\
  \end{tabular}
\end{figure}

\section{Konstanty}

% Popis struktury constant\_pool. Jakým způsobem jsou ukládané konstanty? Jak jsou reprezentované různé typy?
% Jednotlivé typy konstant lze nejspíš popsat jen slovně.
% TODO
% * zmínit run-time constant pool, kdy se symbolické reference nahrazují konkrétními referencemi

Tabulka konstant obsahuje některé číselné konstanty, všechny řetězce a symbolické informace o všech třídách, rozhraních, metodách a členských proměnných, které se v souboru, instrukcích i atributech vyskytují. Tato tabulka se nazývá \textit{constant pool} a slouží v podstatě jako databáze dat, do které se pomocí indexů odkazují další položky souboru. 

Struktura tabulky konstant je popsána pravidlem pro \N{constants}. Symbol \N{constant\_pool\_count} reprezentuje hodnotu $1 + n$, kde $n$ je počet položek v tabulce konstant. Položky tabulky jsou indexované od jedné. Nultý index je vyhrazen pro odkaz na žádnou z položek. Samotná tabulka je reprezentovaná symbolem \N{constant\_pool}.

\begin{figure} [h!]

  \begin{tabular}{r c l}
  \N{constants} &:=& \N{constant\_pool\_count}, \N{constant\_pool}; \\
  \N{constant\_pool\_count} &:=& \N{2B}; \\
  \N{constant\_pool} &:=& \{
          \N{constant\_integer} \\
  &&  $|$ \N{constant\_float} \\
  &&  $|$ \N{constant\_long} \\
  &&  $|$ \N{constant\_double} \\ 
  &&  $|$ \N{constant\_utf8} \\
  &&  $|$ \N{constant\_string} \\ 
  &&  $|$ \N{constant\_nameAndType} \\ 
  &&  $|$ \N{constant\_class} \\
  &&  $|$ \N{constant\_fieldref} \\
  &&  $|$ \N{constant\_methodref} \\
  &&  $|$ \N{constant\_interfaceMethodref} \\
  &&  $|$ \N{constant\_methodHandle} \\ 
  &&  $|$ \N{constant\_methodType} \\
  &&  $|$ \N{constant\_invokeDynamic} \\ 
  &&  \}; \\
  \end{tabular}
\end{figure}

Každá položka tabulky je tvořena označením typu a posloupností bajtů s informacemi o položce. Položky mohou mít různou velikost v závislosti na svém typu a obsahu. Stejným způsobem jsou definovány všechny tabulky v \texttt{class} souboru. Jestliže se jedná o pole, pak prvky pole jsou stejného typu, a proto označení typu v prvcích chybí.

 Číselné konstanty jsou reprezentované symboly \N{constant\_integer}, \N{constant\_float}, \N{constant\_long} a \N{constant\_double} a skládají se jen z typu a číselné hodnoty. 
Symbol \N{constant\_utf8} reprezentuje řetězec v upraveném kódování \texttt{UTF-8} a skládá se z typu, délky pole bajtů a pole bajtů nesoucích reprezentaci řetězce. Znaky řetězce mohou být vhledem ke kódování tvořeny různými počty bajtů. 
Symbol \N{constant\_string} je reprezentací řetězcové konstanty a kromě typu obsahuje index do tabulky konstant na položku \N{constant\_utf8}. 
Třídy a rozhraní jsou reprezentované položkami \N{constant\_class} s odkazy na jejich název (\N{constant\_utf8}). 
Entity jako členské proměnné, metody třídy a metody rozhraní jsou reprezentované položkami \N{constant\_fieldref}, \N{constant\_methodref} a \N{constant\_interfaceMethodref} obsahujícími odkaz na třídu, případně rozhraní, dané entity (\N{constant\_class}), a odkaz na položku se jménem a typem této entity (\N{constant\_nameAndType}). 
Jméno a typ v položce \N{constant\_nameAndType} jsou odkazy na řetězce (\N{constant\_utf8}). 
Položky \N{constant\_methodHandle}, \N{constant\_methodType} a \N{constant\_invokeDynamic} souvisí s podporou dynamických jazyků.

\section{Třída}

% TODO access flags

Každý \texttt{class} soubor reprezentuje jednu třídu nebo rozhraní. Informace o reprezentované entitě jsou definované symbolem \N{class}. Položka \N{this\_class} je odkazem na tuto entitu v tabulce konstant, \N{super\_class} je odkaz na nadřazenou třídu a \N{access\_flags} je bitové pole příznaků pro přístup k této entitě.

\begin{figure} [h!]
  \begin{tabular}{r c l}
  \N{class} &:=& \N{access\_flags}, \N{this\_class}, \N{super\_class};\\
  \N{access\_flags} &:=& \N{2B}; \\
  \N{this\_class} &:=& \N{class\_ref};\\
  \N{super\_class} &:=& \N{class\_ref};\\
  \N{class\_ref} &:=& \N{constant\_pool\_index}; \\
  \N{constant\_pool\_index} &:=& \N{2B}; \\
  \end{tabular}
\end{figure}

Seznam rozhraní, které reprezentovaná třída implementuje, případně reprezentované rozhraní rozšiřuje, je definované v poli \N{interfaces} o \N{interface\_count} prvcích. Prvky jsou odkazy do tabulky konstant na položky \N{constant\_class} reprezentující nějaké rozhraní.

\begin{figure} [h!]
  \begin{tabular}{r c l}
  \N{interface\_list} &:=& \N{interface\_count}, \N{interfaces};\\
  \N{interfaces} &:=& \{ \N{class\_ref} \};\\
  \N{interface\_count} &:=& \N{2B};\\
  \end{tabular}
\end{figure}

\section{Členské proměnné}

% Popis struktury field\_info. Jakým způsobem jsou uložené členské proměnné?
% Popis descriptoru?

Členské proměnné třídy jsou definované v poli členských proměnných \N{fields} o \N{fields\_count} prvcích. Každá proměnná má pole příznaků dané symbolem \N{access\_flags}, jméno dané symbolem \N{name\_ref}, typ daný symbolem \N{descriptor\_ref} a seznam atributů daný symbolem \N{attribute\_list}. Jméno a typ jsou reprezentované odkazem do tabulky konstant na položku \N{utf8\_ref}. Atributům se věnuje kapitola (?).

\begin{figure} [h!]
  \begin{tabular}{r c l}
  \N{field\_list} &:=& \N{fields\_count}, \N{fields};\\
  \N{fields} &:=& \{ \N{field\_info} \};\\
  \N{field\_info} &:=& \N{access\_flags}, \N{name\_ref}, \N{descriptor\_ref}, \N{attribute\_list};\\
  \N{fields\_count} &:=& \N{2B};\\
  \N{name\_ref} &:=& \N{utf8\_ref};\\
  \N{descriptor\_ref} &:=& \N{utf8\_ref};\\
  \N{utf8\_ref} &:=& \N{constant\_pool\_index}; \\
  \end{tabular}
\end{figure}

\section{Metody}

% Popis struktury method\_info. Jakým způsobem jsou uložené metody? Kde jsou uloženy instrukce?
% TODO jen abstraktní metoda nemá code?

Metody reprezentované třídy či rozhraní jsou definované v poli metod \N{methods} o \N{methods\_count} prvcích. Stejně jako u členských proměnných jsou metody popsané bitovým polem příznaků, jménem, typem a seznamem atributů.

\begin{figure} [h!]
  \begin{tabular}{r c l}
  \N{method\_list} &:=& \N{methods\_count}, \N{methods};\\
  \N{methods} &:=& \{ \N{method\_info} \};\\
  \N{method\_info} &:=& \N{access\_flags}, \N{name\_ref}, \N{descriptor\_ref}, \N{attribute\_list};\\
  \N{methods\_count} &:=& \N{2B};\\
  \end{tabular}
\end{figure}

Pokud metoda není abstraktní, pak jedním z jejích atributů je \texttt{Code} s kódem metody.
Tento atribut je definován symbolem \N{code\_attribute}. Položka \N{name\_ref} je odkazem do tabulky konstant na řetězec "Code". Položka \N{attribute\_length} určuje délku atributu v bajtech bez prvních šesti bajtů. Dále položky \N{max\_stack} a \N{max\_locals} označují maximální hloubku operačního zásobníku přepočtenou na jednotku hloubky (?) a maximální počet lokálních proměnných včetně parametrů. Kód metody je reprezentovaným polem bajtů \N{code} o délce \N{code\_length}. Symbol \N{attribute\_list} je seznamem atributů.

\begin{figure} [h!]
  \begin{tabular}{r c l}
  \N{code\_attribute} &:=& \N{name\_ref}, \N{attribute\_length}, \N{code\_info} \\
  \N{code\_info} &:=& \N{max\_stack}, \N{max\_locals}, \N{code\_list}, \N{exception\_list}, \N{attribute\_list}; \\ 
  \N{code\_list} &:=& \N{code\_length}, \N{code} ; \\ 
  \N{code} &:=& \{ \N{B} \}; \\ 
  \N{max\_stack} &:=& \N{2B}; \\ 
  \N{max\_locals} &:=& \N{4B}; \\ 
  \N{code\_length} &:=& \N{4B} ; \\ 
  \end{tabular}
\end{figure}

Informace o zpracování výjimek jsou dostupné v tabulce výjimek \N{exception\_table} o délce \N{exception\_table\_length}. Každá položka tabulky obsahuje dva indexy \N{start\_pc} a \N{end\_pc} do pole \N{code}, které společně definují blok instrukcí, pro které je odchycení dané výjimky aktivní. Dále index \N{handler\_pc} do pole \N{code} odkazující na začátek bloku pro zpracování výjimky. A nakonec index \N{catch\_type} do tabulky konstant na položku \N{constant\_class} reprezentující typ odchycené výjimky. Jestliže je tento index nulový, pak jsou odchytávány všechny výjimky. Na pořadí položek v tabulce výjimek se nevztahují žádná omezení, neboť při odchytávání výjimky se postupuje od nejniternějšího bloku.

% TODO doplnit ukázku kódu a bytekódu

\begin{figure} [h!]
  \begin{tabular}{r c l}
  \N{exception\_list} &:=& \N{exception\_table\_length}, \N{exception\_table} ; \\ 
  \N{exception\_table} &:=& \{ \N{\N{start\_pc}, \N{end\_pc}, \N{handler\_pc}, \N{catch\_type}} \}; \\ 
  \N{start\_pc} &:=& \N{code\_index}; \\ 
  \N{end\_pc} &:=& \N{code\_index}; \\ 
  \N{handler\_pc} &:=& \N{code\_index}; \\ 
  \N{catch\_pc} &:=& \N{class\_ref}; \\ 
  \N{exception\_table\_length} &:=& \N{2B}; \\ 
  \N{code\_index} &:=& \N{2B}; \\
  \end{tabular}
\end{figure}

\section{Instrukce}

% Přehled instrukcí a krátké příklady bajkódu.

\section{Atributy}

% Základní přehled atributů, jen ty podstatné, neboť jich je mnoho.

\begin{figure} [h!]
  \begin{tabular}{r c l}
  \N{attribute\_list} &:=& \N{attributes\_count}, \N{attributes};\\
  \N{attributes} &:=& \{ \N{name\_ref}, \N{attribute\_length}, \N{info} \};\\
  \N{info} &:=& \{ \N{B} \};\\
  \N{attributes\_count} &:=& \N{2B}; \\
  \N{attribute\_length} &:=& \N{4B};\\
  \end{tabular}
\end{figure}


%=========================================================================

%%%%%%%%%%%%%%%%%%%%%%%%%%%%%%%%%%%%%%%%%%%%%%%%%%%%%%%%%%%%%%%%%%%%%%%%%%
\chapter{Nástroje pro manipulaci s~bajtkódem}\label{Tools}

Pro další studium bajtkódu Javy bylo potřeba zvolit vhodný způsob, jakým lze s bajtkódem pracovat.
Vzhledem k tomu, že bajtkód je strojový kód, tak je přímá manipulace prakticky nemožná. Proto je vhodnější využít některý z existujících nástrojů. Tato kapitola je věnovaná třem 
dle ? \cite{} 
nejpoužívanějším knihovnám BCEL, ASM a Javassist. Zvolené knihovny jsou implementované v programovacím jazyce Java a liší se navzájem mírou abstrakce a způsobem manipulace s~bajtkódem.


\section{BCEL}\label{Tools:BCEL}

% Z jakých nejdůležitějších tříd se knihovna skládá?
% Jakým způsobem lze manipulovat s bajtkódem?
% Jaké další nástroje BCEL nabízí?

BCEL \cite{BCEL} nebo-li Byte Code Engineering Library je knihovna, která je součástí projektu Apache Commons. Je poskytovaná pod licencí Apache License 2.0. Poslední verze BCEL 5.2 nepodporuje Javu 8, ale z~repozitáře je dostupná verze 6.0, kde je podpora z~větší části implementovaná. Vývoj knihovny však v~posledních letech není příliš aktivní. 

Programové rozhraní knihovny je dostupné v~balíčku \texttt{org.apache.bcel}. Knihovna obsahuje třídy pro statický popis \texttt{class} souborů, třídy pro dynamické úpravy a vytváření bajtkódu a třídy s~užitečnými nástroji. Syntaktickou analýzu \texttt{class} souboru a vytvoření reprezentace jeho obsahu v~podobě instance třídy \texttt{JavaClass} umožňuje třída \texttt{ClassParser} z~balíčku \texttt{org.apache.bcel.classfile}. Součástí balíčku jsou současně všechny třídy podílející se na popisu obsahu souboru. Pro každou položku souboru je tedy vytvořen nový objekt. Takový přístup může být velmi neefektivní, zejména pokud je třeba zpracovat velké množství souborů. Na druhou stranu třída \texttt{JavaClass} velmi přesně kopíruje formát \texttt{class} souboru tak, jak byl popsán v~kapitole \ref{Bytecode:Format}, včetně tabulky konstant.
Pro dynamické vytváření a úpravu bajtkódu je třeba vyšší míra abstrakce. Tu poskytují třídy z~balíčku \texttt{org.apache.bcel.generic}. Pomocí těchto tříd je třeba sestavit celý obsah \texttt{class} souboru včetně tabulky konstant. Korektnost výsledného bajtkódu lze zkontrolovat třídou \texttt{Verifier}.

% zmínit visitora, BCELLifier, instruction finder

Knihovna BCEL poskytuje pro bajtkód velmi nízkou úroveň abstrakce. Je třeba být seznámen s~formátem \texttt{class} souborů a pracovat s~tabulkou konstant. Bajtkód je navíc reprezentovaný velkým množstvím objektů a neexistuje efektivní způsob, jak zpracovat jen ty informace, které jsou pro danou aplikaci potřeba. Vhodnou alternativou je proto knihovna ASM.


% příklad načtení kódu a výpis 
% příklad vytvoření kódu ?

% FIndBugs

\section{ASM}\label{Tools:ASM}

ASM \cite{ASM} je knihovna od OW2 Consortium poskytovaná pod licencí BSD. Na rozdíl od BCEL se jedná o~aktivní projekt a Java 8 je oficiálně plně podporovaná. ASM si zakládá na snadné použitelnosti, výkonnosti a malé velikosti. 
Knihovna je založena na návrhovém vzoru Návštěvník (Visitor). Místo reprezentace \texttt{class} souboru pomocí objektů jsou při syntaktické analýze volány pro jednotlivé položky metody návštěvníka. Návštěvník může položky zpracovat a předat je dalšímu návštěvníkovi. Pomocí takového zřetězení lze jedním až dvěma průchody \texttt{class} souboru dosáhnout požadovaného zpracování bajtkódu. Pokud je třeba provést větší počet průchodů, může být vhodnější použít objektovou reprezentaci pomocí stromu. ASM umožňuje oba přístupy libovolně kombinovat.

Základní rozhraní je dostupné v~balíčku \texttt{org.objectweb.asm}. Třída \texttt{ClassReader} analyzuje daný \texttt{class} soubor a volá metody návštěvníka, instance třídy rozšiřující abstraktní třídu \texttt{ClassVisitor}. Třída \texttt{ClassVisitor} umožňuje vytvořit sekvenci návštěvníků. Jedním z~těchto návštěvníků může být i instance třídy \texttt{ClassWriter}, která z~parametrů volaných metod vytvoří opět binární reprezentaci bajtkódu. Tato třída může být použita i samostatně pro dynamické generování bajtkódu. Při průchodu souborem i při jeho vytváření je třeba pamatovat na pořadí, ve kterém jsou jednotlivé položky navštíveny.
Programové rozhraní pro objektovou reprezentaci pomocí stromu je v~balíčku \texttt{org.objectweb.asm.tree}. Obsah \texttt{class} souboru je reprezentovaný třídou \texttt{ClassNode}, která tvoří kořen stromu. Jednotlivé položky tvoří uzly. S~takto vytvořeným stromem lze libovolně manipulovat i vytvořit strom zcela nový. Jednotlivé uzly jsou současně návštěvníky daných položek. Díky tomu je možné libovolně přecházet mezi oběma přístupy k~bajtkódu.
Balíčky \texttt{org.objectweb.asm.util}, \texttt{org.objectweb.asm.commons} a \texttt{org.objectweb.asm.tree.analysis} obsahují některé zajímavé nástroje pro zpracování a analýzu bajtkódu.

ASM zaujme svým návrhem a možností výběru mezi dvěma způsoby práce s~bajtkódem. Nabízí vyšší úroveň abstrakce než BCEL, neboť přístup k~tabulce konstant je uživateli zcela odepřen. Na druhou stranu je práce s~bajtkódem stále na úrovni blízké formátu \texttt{class} souboru. Z~popisovaných nástrojů je ASM považován za nejrychlejší.

% ASMFiier plugin

\section{Javassist}\label{Tools:Javassist}

Javassist \cite{Javassist} nebo-li Java Programming Assistant je knihovna poskytovaná pod trojitou licencí MPL, LGPL a Apache License. Je vhodná zejména pro úpravu bajtkódu za běhu programu. Knihovna umožňuje pracovat s~\texttt{class} soubory na dvou úrovních. Úroveň zdrojového kódu nevyžaduje znalost bajtkódu, ale umožňuje s~bajtkódem manipulovat pomocí slovníku programovacího jazyka Java. Úroveň bajtkódu umožňuje přístup k~reprezentaci blízké formátu \texttt{class} souboru. Java 8 je podporovaná.

V~balíčku \texttt{javassist} je dostupné základní rozhraní knihovny. Třída \texttt{CtClass} je reprezentací \texttt{class} souboru. Instanci této třídy je třeba získat z~úložiště reprezentovaného třídou \texttt{ClassPool}. V~tomto úložišti jsou k~dispozici všechny takto načtené třídy. Získanou reprezentaci třídy lze modifikovat a uložit do souboru či pole bajtů, nebo lze vytvořit reprezentovanou třídu. Těla metod lze modifikovat pomocí tříd z~balíčku \texttt{javassist.expr}. Manipulace na úrovni zdrojového kódu má však jistá omezení a nejsou podporovány všechny jazykové konstrukce. Proto balíček \texttt{javassist.bytecode} poskytuje rozhraní pro přímou editaci bajtkódu. Instrukční soubor je zde reprezentovaný třídou \texttt{ClassFile}. K~dispozici je i tabulka konstant reprezentovaná třídou \texttt{ConstPool}.

S~\texttt{class} souborem se v~Javassist opět manipuluje prostřednictvím objektové reprezentace. Zajímavá je však možnost pracovat s~bajtkódem jako s~konstrukcemi programovacího jazyka Java. Javassist tak nabízí mnohem vyšší úroveň abstrakce než BCEL a ASM. Navíc má propracovanější podporu editace bajtkódu za běhu.

%=========================================================================

%%%%%%%%%%%%%%%%%%%%%%%%%%%%%%%%%%%%%%%%%%%%%%%%%%%%%%%%%%%%%%%%%%%%%%%%%%
\chapter{Nástroj pro analýzu bajtkódu}\label{Tool}

% TODO 
% * sumarizující? je to korektní?
% * dostatečně
% * obrázek s instrukcemi a stromem
% * příklady zobecnění instrukcí 
% * základní blok
% * kombinatoriky
% * 

V~této kapitole popisuji návrh a implementaci nástroje \texttt{jbyco}. Nástroj je určen pro zpracování \texttt{class} souborů a získání dat sumarizujících obsahy těchto souborů. Následná analýza dat poslouží k návrhu optimalizace velikosti.

\section{Návrh programu}\label{ToolDesign}

Při návrhu programu jsem zohlednila následující požadavky. 
Výstupy programu by měly umět zodpovědět následující otázky. Jak vypadá obsah konkrétního \texttt{class} souboru? Kolik bajtů, tříd, metod a členských proměnných se zpracovalo? Jaké jsou velikosti jednotlivých položek v souborech? Jaké je využití lokálních proměnných? Jaké jsou typické sekvence instrukcí v souborech? K zodpovězení většiny otázek je vždy třeba zpracovat dostatečně velké množství souborů, aby získané odpovědi byly dostatečně obecné. 

Program jsem vzhledem k požadavkům rozdělila na několik podprogramů, přičemž každý z nich zodpovídá jednu otázku. 
Většinu požadovaných dat lze získat snadno pomocí knihoven ASM a BCEL. Největší problém tak představovalo nalezení typických sekvencí instrukcí.

\subsubsection{Reprezentace sekvencí instrukcí}

Získání typických sekvencí instrukcí vyžaduje uchovávat v paměti všechny různé sekvence různých délek ze všech zpracovaných metod a jejich četnosti výskytu. Každou novou sekvenci je pak třeba porovnat s ostatními, a buď upravit četnost shodné sekvence, nebo vložit novou sekvenci mezi ostatní. To představuje velkou časovou a paměťovou zátěž a nelze zaručit, že program skončí dřív než dojde k nedostatku paměti. Z těchto důvodů bylo třeba vymyslet úspornou datovou strukturu reprezentující sekvence a způsob zotavení se z nedostatku paměti.

Jako vhodná datová struktura se pro sekvence instrukcí nabízí strom. Kořenem stromu by byl prázdný uzel, uzly jednotlivé instrukce a žádný uzel by nesměl mít dva bezprostřední následníky se stejnou instrukcí. Každá cesta z kořene do nějakého uzlu stromu by představovala jednu sekvenci a poslední uzel této cesty by pak obsahoval hodnotu četnosti výskytu sekvence. Do stromu pak stačí vkládat postfixy seznamu instrukcí každé metody, neboť sufixy těchto postfixů tvoří všechny sekvence seznamu instrukcí a každý z těchto sufixů je tvořen cestou z kořene stromu. Stromová reprezentace umožňuje snadné porovnávání a přidávání sekvencí a šetří pamětí, neboť sekvence mohou sdílet společné prefixy.

Nedostatku paměti je třeba předcházet v každém případě. Možným řešením je pravidelně kontrolovat, kolik procent z dostupné paměti se již využilo, a při překročení určité hodnoty zmenšit velikost vytvořeného stromu. Zmenšení stromu je možné dosáhnout jedním průchodem, při kterém se odstraní všechny hrany do uzlů z nižší hodnotou četnosti výskytu než je stanovený práh. Tento práh se následně zvedne. V krajním případě bude strom po ukončení výpočtu obsahovat pouze svůj kořen, ale výpočet neskončí nedostatkem paměti.

\subsubsection{Zobecnění instrukcí a jejich sekvencí}

Každá instrukce je daná svým operačním kódem a parametry. Zkoumat typické sekvence instrukcí s konkrétními hodnotami parametrů může přinést zajímavé výsledky, ale nenalezne obecné typické konstrukce. Instrukce lze tak nahradit jejich zobecněnými protějšky. Zobecnění instrukce se skládá ze svou částí: zobecnění operace a zobecnění parametrů. Zobecnění operačního kódu lze dosáhnout odstraněním typové informace z názvu operace. Zobecnění parametrů může mít dvě úrovně. Na nejnižší úrovni je parametr zastoupený jen svým typem. Na vyšší úrovni je každé hodnotě parametru přiřazen typ a číselný identifikátor. Pokud se taková hodnota vyskytuje v jedné sekvenci vícekrát, přiřazený identifikátor se nemění. Lze tak obecně zkoumat práci s opakujícími se hodnotami parametrů.

V některých případech může být užitečné naopak rozšířit informace o instrukcích. Pracuje-li instrukce s lokální proměnnou a proměnná je jedním z parametrů metody, pak je vhodné nahradit označení proměnné klíčkovým slovem \texttt{this}, nebo identifikátorem parametru metody. Díky tomu je možné pozorovat, jak se v metodě pracuje s jejími parametry a jak třída pracuje se svými metodami a členskými proměnnými.

Seznam instrukcí je vhodné doplnit o návěští, která budou označovat místa skoků. Budou-li tato návěští součástí zkoumaných sekvencí, lze rozlišit jednotlivé základní bloky instrukcí.

Jako možné rozšíření se nabízí práce s divokými kartami.
Divoká karta označuje v sekvenci instrukcí místo, ze kterého byla odebrána alespoň jedna instrukce. Sekvence s divokými kartami tak umožňují zkoumat i typické sekvence instrukcí, které nenásledují bezprostředně za sebou. K vytvoření všech sekvencí s divokými kartami pro danou sekvenci je třeba určit všechny možné intervaly indexů, ve kterých budou odebrány instrukce. Takové intervaly lze určit pomocí kombinatoriky.



\section{Popis implementace}\label{ToolImplementation}


\section{Překlad a spuštění}\label{ToolRun}

%=========================================================================

%%%%%%%%%%%%%%%%%%%%%%%%%%%%%%%%%%%%%%%%%%%%%%%%%%%%%%%%%%%%%%%%%%%%%%%%%%
\chapter{Analýza bajtkódu}

% Popis způsobu analýzy rozsáhlého vzorku testovacích dat, prezentace výsledků a zhodnocení.

\section{Nástroje}

% Zde bude uveden popis nástrojů, které jsem použila k analýze bajtkódu. 
% Jedná se o nástroje, které jsem sama implementovala, ale popis implementace v této práci příliš rozvádět nechci.

\subsubsection{Načtení testovacích dat}

% Jakým způsobem načítám data z class souborů?

\subsubsection{Zpracování bajtkódu}

% Jakým způsobem a pomocí kterého nástroje (ACM) zpracovávám data?

\subsubsection{Nalezení typických sekvencí instrukcí}

% Jak vypadá graf podřetězců? Proč je pro analýzu vhodnější než hash tabulka s podřetězci jako klíči a četnostmi výskytu jako hodnotami?
% Jak lze pomocí grafu podřetězců získat typické sekvence instrukcí? Jak lze graf zobrazit?

\section{Testovací data}

% Jaká data jsem použila k analýze? Odkud jsem je získala? Jakých kritérií jsem se držela?

\section{Výsledky analýzy}

% Zde budou uvedeny tabulky a grafy s výsledky analýzy.

\section{Vyhodnocení}

% Shrnutí výsledků a komentář. Jak mohu výsledky analýzy použít pro další postup práce?

%=========================================================================

%%%%%%%%%%%%%%%%%%%%%%%%%%%%%%%%%%%%%%%%%%%%%%%%%%%%%%%%%%%%%%%%%%%%%%%%%%
\chapter{Závěr}

% Shrnutí toho, co jsem udělala, jakých výsledků jsem dosáhla a jak budu pokračovat.


%=========================================================================


%=========================================================================
 % viz. content.tex

  % Pouzita literatura
  % ----------------------------------------------
\ifczech
  \makeatletter
  \def\@openbib@code{\addcontentsline{toc}{chapter}{Literatura}}
  \makeatother
  \bibliographystyle{czechiso}
\else 
  \makeatletter
  \def\@openbib@code{\addcontentsline{toc}{chapter}{Literature}}
  \makeatother
  \bibliographystyle{plain}
%  \bibliographystyle{alpha}
\fi
  \begin{flushleft}
  \bibliography{literature} % viz. literature.bib
  \end{flushleft}

  % Prilohy
  % ---------------------------------------------
  \appendix
\ifczech
  \renewcommand{\appendixpagename}{Přílohy}
  \renewcommand{\appendixtocname}{Přílohy}
  \renewcommand{\appendixname}{Příloha}
\fi
  \appendixpage
\ifczech
  \section*{Seznam příloh}
  \addcontentsline{toc}{section}{Seznam příloh}
\else
  \section*{List of Appendices}
  \addcontentsline{toc}{section}{List of Appendices}
\fi
  \startcontents[chapters]
  \printcontents[chapters]{l}{0}{\setcounter{tocdepth}{2}}
  \chapter{Obsah CD}

Přiložené CD obsahuje:
\begin{itemize*}
\item písemnou zprávu ve formátu PDF,
\item zdrojové texty písemné zprávy,
\item zdrojové soubory programů,
\item spustitelné soubory programů,
\item manuál.
\end{itemize*}

%\chapter{Manual}
%\chapter{Konfigrační soubor}
%\chapter{RelaxNG Schéma konfiguračního soboru}
%\chapter{Plakat}


 % viz. appendix.tex
\end{document}
