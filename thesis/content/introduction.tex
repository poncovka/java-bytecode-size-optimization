%%%%%%%%%%%%%%%%%%%%%%%%%%%%%%%%%%%%%%%%%%%%%%%%%%%%%%%%%%%%%%%%%%%%%%%%%%
\chapter{Úvod}\label{Introduction}

Kód programu je obvykle optimalizován s~cílem minimalizovat dobu běhu programu, ale hlavním požadavkem může být i kratší kód či efektivnější práce s~dostupnými prostředky, jak uvádí Aho \cite{Aho:Compilers}. Zatímco optimalizace rychlosti je u~jazyka Java dobře zpracovaným tématem \cite{Kazi:Performance}, optimalizace velikosti kódu se řeší především v~souvislosti s~vestavěnými systémy \cite{Clausen:Embedded} a obfuskací \cite{Chan:Obfuscation}. Ve vestavěných systémech se však používají specializované edice Javy a obfuskace kódu vede k~modifikaci struktury programu. Kratší kód přitom zabírá méně paměti, rychleji se přenáší po síti a může vést i k~rychlejšímu běhu programu. Cílem této práce je proto studium bajtkódu Javy SE z~hlediska jeho velikosti a návrh metod pro optimalizaci velikosti bajtkódu. Výstupem práce jsou nástroje \texttt{jbyca} a \texttt{jbyco} pro analýzu a optimalizaci bajtkódu.

V~kapitole \ref{Bytecode} se věnuji obecné speficikaci bajtkódu Javy. Popisuji virtuální stroj Java Virtual Machine, způsob, jakým je bajtkód interpretován, a zabývám se formátem, v~jakém je bajtkód uložen v~instrukčních souborech. V~kapitole \ref{Tools} uvádím stručný popis existujících nástrojů pro manipulaci s~bajtkódem a shrnuji jejich výhody a nevýhody. Konkrétně zmiňuji BCEL, ASM a Javassist. Tyto nástroje jsem využila při návrhu a implementaci nástroje \texttt{jbyca} pro analýzu bajtkódu popsaného v~kapitole \ref{Jbyca}. Nástroj jsem aplikovala na vybraný vzorek dat a získané výstupy zpracovala a vyhodnotila v~kapitole \ref{Optimization}. Na základě výsledků jsem navrhla metody pro optimalizaci velikosti a implementovala je v~nástroji \texttt{jbyco}, kterému je věnovaná kapitola \ref{Jbyco}. Nakonec jsem s~užitím tohoto nástroje optimalizovala vzorová data a vyhodnotila účinky optimalizačních metod.


%=========================================================================

