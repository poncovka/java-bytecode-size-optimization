%%%%%%%%%%%%%%%%%%%%%%%%%%%%%%%%%%%%%%%%%%%%%%%%%%%%%%%%%%%%%%%%%%%%%%%%%%
\chapter{Úvod}\label{Introduction}

% Úvod do tématu. 
TODO: Úvod do tématu, přehled kapitol.

%V~této práci se zabývám bajtkódem Javy z~hlediska optimalizace jeho velikosti. V~kapitole~\ref{JVM} popisuji virtuální stroj Java Virtual Machine a způsob, jakým je bajtkód interpretován. Kapitola \ref{Format} je věnovaná formátu, v~jakém je bajtkód uložen v~instrukčních souborech. Dále uvádím stručný popis nástrojů pro manipulaci s~bajtkódem a shrnuji jejich výhody a nevýhody v~kapitole \ref{Tools}. Konkrétně zmiňuji nástroje BCEL, ACM a Javassist. Kapitola \ref{Tool} je věnovaná návrhu a implementaci nástroje pro analýzu bajtkódu. Pomocí tohoto nástroje jsem získala data, která jsem zpracovala a vyhodnotila v~kapitole \ref{Analysis}. Cílem této práce je seznámení se s~bajtkódem a diagnostika míst vhodných k~optimalizaci z~hlediska velikosti.

%=========================================================================

