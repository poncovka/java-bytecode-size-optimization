%%%%%%%%%%%%%%%%%%%%%%%%%%%%%%%%%%%%%%%%%%%%%%%%%%%%%%%%%%%%%%%%%%%%%%%%%%
\chapter{Nástroj pro analýzu bajtkódu}\label{Tool}

V~této kapitole popisuji návrh a implementaci nástroje \texttt{jbyco}. Nástroj je určen pro zpracování velkého množství souborů a získání dat, která jsou vhodná pro analýzu bajtkódu.

\section{Analýza problému a návrh řešení}\label{ToolDesign}

Pro analýzu bajtkódu jsem potřebovala získat data, která by vhodným způsobem reprezentovala analyzovaný bajtkód. Zajímaly mne celkové součty položek v~\texttt{class} souboru, využití lokálních proměnných a parametrů a typické sekvence instrukcí. Dále jsem potřebovala vyřešit načítání velkého množství vstupních dat.

U~hledání typických sekvencí jsem zvažovala jednotlivé sekvence a jejich četnosti zaznamenávat jednoduše pomocí tabulky klíč-hodnota. Takový přístup mi nepřipadal vhodný z~hlediska paměťové složitosti, neboť by to znamenalo udržovat v~paměti všechny podsekvence sekvencí instrukcí. Rozhodla jsem se proto pro reprezentaci sekvencí pomocí orientovaného acyklického grafu, tzv. grafu sufixů. Na instrukce se lze dívat jako na prvky abecedy a na sekvence instrukcí jako na řetězce. Pak definované cesty v~grafu sufixů tvoří sufixy reprezentovaných řetězců. Prefixy těchto sufixů tvoří všechny podřetězce reprezentovaných řetězců. 

Graf sufixů se skládá z~kořene a uzlů. Všechny uzly kromě kořene reprezentují prvky abecedy a pro každou hranu je definovaná množina cest, které danou hranou prochází. Každá taková cesta v~grafu má vlastní čítač, který určuje, kolik stejných cest daná cesta reprezentuje. Když se do grafu přidává další sufix, postupuje se grafem směrem od kořenu. Pokud je aktuální prvek sufixu stejný jako prvek některého ze sousedů aktuálního uzlu, pak se daný uzel stane aktuálním a začne se zpracovávat následující prvek sufixu. Pokud takový soused neexistuje, vybere se uzel se stejným prvkem, který je nedosažitelný z~aktuálního uzlu. K~tomuto uzlu se vytvoří z~aktuálního uzlu hrana, vytvoří se nová cesta a uzel se označí za aktuální. Jestliže takový uzel neexistuje, pak se vytvoří nový uzel, hrana i cesta a nový uzel se stane aktuálním. Po vložení posledního prvku sufixu se inkrementuje čítač cesty, kterou sufix duplikuje, nebo se vytvoří cesta nová, pokud už vytvořena nebyla. Cílem je v~podstatě minimalizovat duplicitu podřetězců v~grafu.

V~takto vytvořeném grafu sufixů lze následně zjednodušit každou hranu, kde je součet čítačů všech cest menší než daná hodnota. Prvek uzlu, do kterého vede taková hrana, lze nahradit tzv. divokou kartou a sekvenčně i paralelně sousedící uzly s~divokou kartou lze sloučit do jednoho uzlu. V~takto zjednodušeném grafu lze nalézt typické vzory sekvencí instrukcí.

\section{Popis implementace}\label{ToolImplementation}

Nástroj jsem implementovala v~jazyce Java s~pomocí knihovny BCEL. Hlavní metoda \texttt{main} je součástí třídy \texttt{App} v~balíčku \texttt{jbyco}. V~této metodě se zpracují parametry, vytvoří se iterátor vstupních souborů a spustí daná analýza. Iterátor vstupních souborů je reprezentovaný třídou \texttt{BytecodeFiles} z~balíčku \texttt{jbyco.io}. Balíček obsahuje třídy pro různé vstupně-výstupní operace. Třída  \texttt{BytecodeFiles} rekurzivně prochází všechny soubory, složky a \texttt{jar} soubory a vrací instance třídy \texttt{BytecodeFile} reprezentující \texttt{class} soubory. Rozhraní \texttt{Analyzer} z~balíčku \texttt{jbyco.analyze} popisuje rozhraní tříd, které provádí analýzu. V~balíčku \texttt{jbyco.analyze.size} jsou třídy potřebné k~analýze velikosti, v~balíčku \texttt{jbyco.analyze.locals} třídy určené k~analýze využití lokálních proměnných a parametrů a v~balíčku
\texttt{jbyco.analyze.patterns} třídy pro nalezení typických sekvencí instrukcí. Výstupem každé analýzy je vytištěná tabulka se zjištěnými daty. Třídy pro práci s~grafem sufixů jsou obsaženy v~balíčku \texttt{jbyco.analyze.patterns.graph}.
Překlad nástroje lze provést příkazem \texttt{gradle build}. Aplikaci lze spustit příkazem \texttt{gradle run -Dmyargs="$args$"}. Argument \texttt{help} vypíše nápovědu k~programu.

Analýza velikosti a analýza využití proměnných běží i pro velké množství souborů velmi rychle, zatímco vyhledávání typických vzorů je velmi pomalé. Při bližším zkoumání jsem odhalila, že nejvíce času program stráví zajišťováním acykličnosti grafu. To, že graf není strom, je zajímavé z~hlediska úspory paměti i zjednodušování hran. Cenou je však čas běhu programu. Budu muset zvážit volbu jiné reprezentace. Algoritmus pro zjednodušování hran jsem implementovala, ale není zcela odladěný, proto jsem jej pro analýzu nepoužila.


%=========================================================================
