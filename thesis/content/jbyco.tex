%%%%%%%%%%%%%%%%%%%%%%%%%%%%%%%%%%%%%%%%%%%%%%%%%%%%%%%%%%%%%%%%%%%%%%%%%%
\chapter{Nástroj pro analýzu bajtkódu}\label{Tool}

% TODO 
% * sumarizující? je to korektní?
% * dostatečně
% * obrázek s instrukcemi a stromem
% * příklady zobecnění instrukcí 
% * základní blok
% * kombinatoriky
% * abstrahování - zobecnění
% * kešování


V~této kapitole popisuji návrh a implementaci nástroje \texttt{jbyco}. Nástroj je určen pro zpracování \texttt{class} souborů a získání dat sumarizujících obsahy těchto souborů. Následná analýza dat poslouží k návrhu optimalizace velikosti.

\section{Návrh programu}\label{ToolDesign}

Při návrhu programu jsem zohlednila následující požadavky. 
Výstupy programu by měly umět zodpovědět následující otázky. Jak vypadá obsah konkrétního \texttt{class} souboru? Kolik bajtů, tříd, metod a členských proměnných se zpracovalo? Jaké jsou velikosti jednotlivých položek v souborech? Jaké je využití lokálních proměnných? Jaké jsou typické sekvence instrukcí v souborech? K zodpovězení většiny otázek je vždy třeba zpracovat dostatečně velké množství souborů, aby získané odpovědi byly dostatečně obecné. 

Program jsem vzhledem k požadavkům rozdělila na několik podprogramů, přičemž každý z nich zodpovídá jednu otázku. 
Většinu požadovaných dat lze získat snadno pomocí knihoven ASM a BCEL. Největší problém tak představovalo nalezení typických sekvencí instrukcí.

\subsubsection{Reprezentace sekvencí instrukcí}

Získání typických sekvencí instrukcí vyžaduje uchovávat v paměti všechny různé sekvence různých délek ze všech zpracovaných metod a jejich četnosti výskytu. Každou novou sekvenci je pak třeba porovnat s ostatními, a buď upravit četnost shodné sekvence, nebo vložit novou sekvenci mezi ostatní. To představuje velkou časovou a paměťovou zátěž a nelze zaručit, že program skončí dřív než dojde k nedostatku paměti. Z těchto důvodů bylo třeba vymyslet úspornou datovou strukturu reprezentující sekvence a způsob zotavení se z nedostatku paměti.

Jako vhodná datová struktura se pro sekvence instrukcí nabízí strom. Kořenem stromu by byl prázdný uzel, uzly jednotlivé instrukce a žádný uzel by nesměl mít dva bezprostřední následníky se stejnou instrukcí. Každá cesta z kořene do nějakého uzlu stromu by představovala jednu sekvenci a poslední uzel této cesty by pak obsahoval hodnotu četnosti výskytu sekvence. Do stromu pak stačí vkládat postfixy seznamu instrukcí každé metody, neboť sufixy těchto postfixů tvoří všechny sekvence seznamu instrukcí a každý z těchto sufixů je tvořen cestou z kořene stromu. Stromová reprezentace umožňuje snadné porovnávání a přidávání sekvencí a šetří pamětí, neboť sekvence mohou sdílet společné prefixy.

Nedostatku paměti je třeba předcházet v každém případě. Možným řešením je pravidelně kontrolovat, kolik procent z dostupné paměti se již využilo, a při překročení určité hodnoty zmenšit velikost vytvořeného stromu. Zmenšení stromu je možné dosáhnout jedním průchodem, při kterém se odstraní všechny hrany do uzlů z nižší hodnotou četnosti výskytu než je stanovený práh. Tento práh se následně zvedne. V krajním případě bude strom po ukončení výpočtu obsahovat pouze svůj kořen, ale výpočet neskončí nedostatkem paměti.

\subsubsection{Zobecnění instrukcí a jejich sekvencí}

Každá instrukce je daná svým operačním kódem a parametry. Zkoumat typické sekvence instrukcí s konkrétními hodnotami parametrů může přinést zajímavé výsledky, ale nenalezne obecné typické konstrukce. Instrukce lze tak nahradit jejich zobecněnými protějšky. Zobecnění instrukce se skládá ze svou částí: zobecnění operace a zobecnění parametrů. Zobecnění operačního kódu lze dosáhnout odstraněním typové informace z názvu operace. Zobecnění parametrů může mít dvě úrovně. Na nejnižší úrovni je parametr zastoupený jen svým typem. Na vyšší úrovni je každé hodnotě parametru přiřazen typ a číselný identifikátor. Pokud se taková hodnota vyskytuje v jedné sekvenci vícekrát, přiřazený identifikátor se nemění. Lze tak obecně zkoumat práci s opakujícími se hodnotami parametrů.

V některých případech může být užitečné naopak rozšířit informace o instrukcích. Pracuje-li instrukce s lokální proměnnou a proměnná je jedním z parametrů metody, pak je vhodné nahradit označení proměnné klíčkovým slovem \texttt{this}, nebo identifikátorem parametru metody. Díky tomu je možné pozorovat, jak se v metodě pracuje s jejími parametry a jak třída pracuje se svými metodami a členskými proměnnými.

Seznam instrukcí je vhodné doplnit o návěští, která budou označovat místa skoků. Budou-li tato návěští součástí zkoumaných sekvencí, lze rozlišit jednotlivé základní bloky instrukcí.

Jako možné rozšíření se nabízí práce s divokými kartami.
Divoká karta označuje v sekvenci instrukcí místo, ze kterého byla odebrána alespoň jedna instrukce. Sekvence s divokými kartami tak umožňují zkoumat i typické sekvence instrukcí, které nenásledují bezprostředně za sebou. K vytvoření všech sekvencí s divokými kartami pro danou sekvenci je třeba určit všechny možné intervaly indexů, ve kterých budou odebrány instrukce. Takové intervaly lze určit pomocí kombinatoriky.

\section{Popis implementace}\label{ToolImplementation}

Program jsem implementovala v jazyce Java 8 s použitím knihoven ASM a BCEL a nástroje Gradle. V hlavním balíčku \texttt{jbyco} je třída \texttt{App} s metodou \texttt{main}, která dle předaných parametrů spustí některý z nástrojů. Každý nástroj má definovanou vlastní metodu \texttt{main}, lze jej tedy spustit i samostatně.

Balíček \texttt{jbyco.lib} slouží jako knihovna užitečných tříd a funkcí. Jeho součástí jsou třídy \texttt{AbstractOption} a \texttt{AbstractOptions} sloužící k definování a zpracování argumentů příkazové řádky a třída \texttt{Utils} obsahující různé statické metody. 

Balíček \texttt{jbyco.io} obsahuje třídy pro práci se soubory. Třída \texttt{BytecodeFiles} umožňuje iterovat přes všechny \texttt{class} soubory, nalezené v daném adresáři, jeho podadresářích a \texttt{jar} souborech. Iterátor vrací instance třídy implementující rozhraní \texttt{BytecodeFile} z balíčku \texttt{jbyco.io.files}. V tomto balíčku jsou pomocné třídy pro iteraci a reprezentace souborů a adresářů. Třída \texttt{BytecodePrinter} je nástroj pro vypsání obsahu \texttt{class} souboru. Soubor je vypsán prostřednictvím třídy \texttt{BytecodeWriter} a jeho tabulka konstant pomocí třídy \texttt{ConstantPoolWriter}.  

Nástroje pro analýzu bajtkódu jsou umístěné v balíčku \texttt{jbyco.analyze}. V rozhraní \texttt{Analyzer} jsou deklarované metody pro zpracování \texttt{BytecodeFile} souboru (\texttt{processFile}) a výpis získaných dat (\texttt{print}). Jednotlivé nástroje pro analýzu jsou dostupné v čtyřech balíčcích.

\subsubsection{Balíček \texttt{jbyco.analyze.statictics}}

Nástroj pro získání souhrnných informací o zpracovaných \texttt{class} souborech je implementovaný ve třídě \texttt{StatisticsCollector}. Nástroj prochází jednotlivé položky souboru a  počet těchto položek aktualizuje v instanci třídy \texttt{StatisticsMap}. Po zpracování všech souborů se vypíše tabulka s typem položky, celkovým počtem výskytu tohoto typu a počtem výskytu přepočteným na jeden soubor. 

\subsubsection{Balíček \texttt{jbyco.analyze.size}}

Přehled velikostí jednotlivých položek v souborech poskytuje nástroj \texttt{SizeAnalyzer}. Nástroj určuje velikosti zpracovávaných položek a informace o jejich velikostech udržuje v instanci třídy \texttt{SizeMap}. Výstupem je tabulka s typem položky, celkovou velikostí tohoto typu v souborech, průměrnou velikostí pro jednu položku a velikostí ku celkové velikosti zpracovaných souborů. 

\subsubsection{Balíček \texttt{jbyco.analyze.locals}}

Data o využití lokálních proměnných jsou dostupná s nástrojem \texttt{LocalsAnalyzer}. Nástroj s pomocí třídy \texttt{LocalsMap} zaznamenává pro každou metodu počet formálních parametrů a lokálních proměnných a jejich použití v instrukcích dané metody. Výstupem tohoto nástroje jsou dvě tabulky: jedna pro lokální proměnné a jedna pro formální parametry metody. Každá z tabulek obsahuje index do tabulky lokálních proměnných, počet metod, které s danou proměnnou pracují, počet instrukcí pro načítání, ukládání a inkrementaci, celkový počet instrukcí, které používají danou proměnnou a hodnoty přepočtené na jednu metodu.

\subsubsection{Balíček \texttt{jbyco.analyze.patterns}}

Nástroj \texttt{PatternsAnalyzer} umožňuje vyhledat typické sekvence instrukcí. Dle návrhu potřebuje ke své práci stromovou reprezentaci sekvencí a abstrahovanou reprezentaci instrukcí. K těmto účelům slouží následující balíčky.

V balíčku \texttt{jbyco.analyze.patterns.tree} jsou dostupné třídy pro stromovou reprezentaci. Uzel stromu je reprezentován třídou \texttt{Node} a nese položku a čítač četnosti výskytu této položky. Třída \texttt{Tree} reprezentuje strom a obsahuje kořenový uzel s položkou \texttt{Root}. Budování stromu, vkládání sekvencí a ořezávání stromu zajišťuje třída \texttt{TreeBuilder}. Výpis všech cest ve stromu a jejich četností umožňuje třída \texttt{PathsWriter}.

Balíček \texttt{jbyco.analyze.patterns.instructions} obsahuje rozhraní a třídy pro reprezentaci a abstrakci instrukcí. Abstrahování instrukcí umožňuje třída \texttt{Abstractor}. Třída implementuje rozhraní návštěvníka metody z knihovny ASM. Jednotlivé instrukce jsou abstrahovány a ukládány do seznamu instrukcí typu \texttt{AbstractInstruction}. Abstrakce instrukce spočívá v abstrahování operačního kódu, parametrů a návěští. Instance tříd pro abstrakci jednotlivých komponent jsou předány třídě \texttt{Abstractor} v konstruktoru. Z abstrahovaných komponent lze sestavit abstrahovanou instrukci typu \texttt{Instruction}. Součástí balíčku jsou i třídy \texttt{Cache} a \texttt{CachedInstruction}. Třída \texttt{Cache} spravuje mapu slabých referencí na abstraktní instrukce mapovaných na slabé reference instrukcí typu \texttt{CachedInstruction}. Pro každou abstrahovanou instrukci pak vrátí odpovídající instrukci z mapy. Pro různé objekty se shodnými instrukcemi tak \texttt{Cache} vrátí stejný objekt. Třídy slouží k úspoře paměti.

V balíčku \texttt{jbyco.analyze.patterns.labels} jsou rozhraní a třídy pro práci s návěštími. Abstrahované návěští je reprezentované rozhraním \texttt{AbstractLabel}. Abstrakci návěští popisuje rozhraní \texttt{AbstractLabelFactory}. Implementacemi jsou třídy \texttt{NumberedLabel} a \texttt{NumberedLabelFactory}, které každému návěští přiřazují relativní identifikátor. Znamená to, že první návěští v dané sekvenci bude mít identifikátor 1 a druhé návěští identifikátor 1, pokud je shodné s prvním, jinak 2. Třída \texttt{ActiveLabelsFinder} implementuje návštěvníka metody z knihovny ASM a slouží k získání množiny návěští, na které se skáče z nějaké instrukce dané metody. Taková návěští dále označuji za aktivní.

S operacemi se pracuje v balíčku \texttt{jbyco.analyze.patterns.operations}. Operace jsou definované pomocí výčtů implementujících rozhraní \texttt{AbstractOperation}. Každá operace reprezentuje skupinu operačních kódů. Výčet \texttt{GeneralOperation} obsahuje operace bez typových informací na rozdíl od výčtu \texttt{TypedOperation}. Výčty \texttt{GeneralHandleOperation} a \texttt{TypedHandleOperation} implementující rozhraní \texttt{AbstractHandleOperation} definují operace, které se mohou vyskytovat v položkách \textit{constant\_methodHandle}. Rozhraní pro abstrakci operačních kódů má název \texttt{AbstractOperationFactory}. Implementacemi tohoto rozhraní jsou třídy \texttt{GeneralOperationFactory} a \texttt{TypedOperationFactory}. 

Balíček \texttt{jbyco.analyze.patterns.parameters} umožňuje práci s parametry instrukcí. Každá třída reprezentující parametr implementuje rozhraní \texttt{AbstractParameter}. Těmito implementacemi jsou výčty \texttt{ParameterType} popisující typ parametru a \texttt{ParameterValue} sloužící k popisu hodnot \texttt{null} a \texttt{this}, třída \texttt{NumberedParameter} popisující parametr pomocí typu a relativního identifikátoru a třída \texttt{FullParameter} sestávající z typu parametru a pole hodnot. Abstrahování je deklarované v rozhraní \texttt{AbstractParameterFactory} a implementované v následujících třídách. Metody třídy \texttt{GeneralParameterFactory} vrací instance z výčtu \texttt{ParameterType}, metody třídy \texttt{NumberedParameterFactory} vrací instance třídy \texttt{NumberedParameter} a metody třídy \texttt{FullParameterFactory} vrací instance třídy \texttt{FullParameter}.

Třídy balíčku \texttt{jbyco.analyze.patterns.wild} slouží k vytváření sekvencí s divokými kartami. Divoká karta je reprezentovaná hodnotou \texttt{null}. Třída \texttt{Combination} je iterátorem přes všechny možné kombinace intervalů v sekvenci, které budou nahrazeny divokými kartami. Třída \texttt{WildSequenceGenerator} je iterátorem přes všechny sekvence s divokými kartami, které lze z dané sekvence vytvořit.

Hledání typických sekvencí probíhá ve třídě \texttt{PatternsAnalyzer} následovně. Každý \texttt{class} soubor je převeden na stromovou reprezentaci knihovny ASM. Každou metodu navštíví instance třídy \texttt{ActiveLabelsFinder} a vyhledá všechna aktivní návěští. Návěští, která nejsou aktivní budou při dalším zpracování vynechána. Ze seznamu instrukcí metody se postupně vygenerují všechny sekvence instrukcí dané délky. Z těchto sekvencí mohou být dále generované sekvence s daným počtem divokých karet. Každá vygenerovaná sekvence je pomocí třídy \texttt{Abstractor} převedena na sekvenci abstrahovaných instrukcí typu \texttt{AbstractInstruction}. Abstrahované instrukce jsou prostřednictvím třídy \texttt{Cache} nahrazeny instrukcemi typu \texttt{CachedInstruction}. Výsledná sekvence je předaná instanci třídy \texttt{TreeBuilder}, která sekvenci vloží do vytvářeného stromu. Po zpracování každého souboru tato třída zkontroluje dostupnou paměť a případně provede ořezání stromu. Nakonec se pomocí třídy \texttt{PathsWriter} vytiskne obsah stromu. Výstupem jsou nalezené sekvence a jejich četnosti výskytu. Výstup není seřazený.

\section{Překlad a spuštění}\label{ToolRun}



%=========================================================================
