%%%%%%%%%%%%%%%%%%%%%%%%%%%%%%%%%%%%%%%%%%%%%%%%%%%%%%%%%%%%%%%%%%%%%%%%%%
\chapter{Nástroj pro analýzu bajtkódu}\label{Tool}

% TODO 
% * sumarizující? je to korektní?
% * dostatečně
% * obrázek s instrukcemi a stromem
% * příklady zobecnění instrukcí 
% * základní blok
% * kombinatoriky
% * 

V~této kapitole popisuji návrh a implementaci nástroje \texttt{jbyco}. Nástroj je určen pro zpracování \texttt{class} souborů a získání dat sumarizujících obsahy těchto souborů. Následná analýza dat poslouží k návrhu optimalizace velikosti.

\section{Návrh programu}\label{ToolDesign}

Při návrhu programu jsem zohlednila následující požadavky. 
Výstupy programu by měly umět zodpovědět následující otázky. Jak vypadá obsah konkrétního \texttt{class} souboru? Kolik bajtů, tříd, metod a členských proměnných se zpracovalo? Jaké jsou velikosti jednotlivých položek v souborech? Jaké je využití lokálních proměnných? Jaké jsou typické sekvence instrukcí v souborech? K zodpovězení většiny otázek je vždy třeba zpracovat dostatečně velké množství souborů, aby získané odpovědi byly dostatečně obecné. 

Program jsem vzhledem k požadavkům rozdělila na několik podprogramů, přičemž každý z nich zodpovídá jednu otázku. 
Většinu požadovaných dat lze získat snadno pomocí knihoven ASM a BCEL. Největší problém tak představovalo nalezení typických sekvencí instrukcí.

\subsubsection{Reprezentace sekvencí instrukcí}

Získání typických sekvencí instrukcí vyžaduje uchovávat v paměti všechny různé sekvence různých délek ze všech zpracovaných metod a jejich četnosti výskytu. Každou novou sekvenci je pak třeba porovnat s ostatními, a buď upravit četnost shodné sekvence, nebo vložit novou sekvenci mezi ostatní. To představuje velkou časovou a paměťovou zátěž a nelze zaručit, že program skončí dřív než dojde k nedostatku paměti. Z těchto důvodů bylo třeba vymyslet úspornou datovou strukturu reprezentující sekvence a způsob zotavení se z nedostatku paměti.

Jako vhodná datová struktura se pro sekvence instrukcí nabízí strom. Kořenem stromu by byl prázdný uzel, uzly jednotlivé instrukce a žádný uzel by nesměl mít dva bezprostřední následníky se stejnou instrukcí. Každá cesta z kořene do nějakého uzlu stromu by představovala jednu sekvenci a poslední uzel této cesty by pak obsahoval hodnotu četnosti výskytu sekvence. Do stromu pak stačí vkládat postfixy seznamu instrukcí každé metody, neboť sufixy těchto postfixů tvoří všechny sekvence seznamu instrukcí a každý z těchto sufixů je tvořen cestou z kořene stromu. Stromová reprezentace umožňuje snadné porovnávání a přidávání sekvencí a šetří pamětí, neboť sekvence mohou sdílet společné prefixy.

Nedostatku paměti je třeba předcházet v každém případě. Možným řešením je pravidelně kontrolovat, kolik procent z dostupné paměti se již využilo, a při překročení určité hodnoty zmenšit velikost vytvořeného stromu. Zmenšení stromu je možné dosáhnout jedním průchodem, při kterém se odstraní všechny hrany do uzlů z nižší hodnotou četnosti výskytu než je stanovený práh. Tento práh se následně zvedne. V krajním případě bude strom po ukončení výpočtu obsahovat pouze svůj kořen, ale výpočet neskončí nedostatkem paměti.

\subsubsection{Zobecnění instrukcí a jejich sekvencí}

Každá instrukce je daná svým operačním kódem a parametry. Zkoumat typické sekvence instrukcí s konkrétními hodnotami parametrů může přinést zajímavé výsledky, ale nenalezne obecné typické konstrukce. Instrukce lze tak nahradit jejich zobecněnými protějšky. Zobecnění instrukce se skládá ze svou částí: zobecnění operace a zobecnění parametrů. Zobecnění operačního kódu lze dosáhnout odstraněním typové informace z názvu operace. Zobecnění parametrů může mít dvě úrovně. Na nejnižší úrovni je parametr zastoupený jen svým typem. Na vyšší úrovni je každé hodnotě parametru přiřazen typ a číselný identifikátor. Pokud se taková hodnota vyskytuje v jedné sekvenci vícekrát, přiřazený identifikátor se nemění. Lze tak obecně zkoumat práci s opakujícími se hodnotami parametrů.

V některých případech může být užitečné naopak rozšířit informace o instrukcích. Pracuje-li instrukce s lokální proměnnou a proměnná je jedním z parametrů metody, pak je vhodné nahradit označení proměnné klíčkovým slovem \texttt{this}, nebo identifikátorem parametru metody. Díky tomu je možné pozorovat, jak se v metodě pracuje s jejími parametry a jak třída pracuje se svými metodami a členskými proměnnými.

Seznam instrukcí je vhodné doplnit o návěští, která budou označovat místa skoků. Budou-li tato návěští součástí zkoumaných sekvencí, lze rozlišit jednotlivé základní bloky instrukcí.

Jako možné rozšíření se nabízí práce s divokými kartami.
Divoká karta označuje v sekvenci instrukcí místo, ze kterého byla odebrána alespoň jedna instrukce. Sekvence s divokými kartami tak umožňují zkoumat i typické sekvence instrukcí, které nenásledují bezprostředně za sebou. K vytvoření všech sekvencí s divokými kartami pro danou sekvenci je třeba určit všechny možné intervaly indexů, ve kterých budou odebrány instrukce. Takové intervaly lze určit pomocí kombinatoriky.



\section{Popis implementace}\label{ToolImplementation}


\section{Překlad a spuštění}\label{ToolRun}

%=========================================================================
