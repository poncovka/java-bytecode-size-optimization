%%%%%%%%%%%%%%%%%%%%%%%%%%%%%%%%%%%%%%%%%%%%%%%%%%%%%%%%%%%%%%%%%%%%%%%%%%
\chapter{Java Virtual Machine}\label{JVM}

Architektura Javy se podle Vennerse \cite{Venners:InsideJVM} skládá z~programovacího jazyka Java, formátu instrukčního souboru, aplikačního programového rozhraní Java Application Programming Interface (Java API) a virtuálního stroje Java Virtual Machine (JVM). Pro psaní zdrojových kódů a jejich spouštění je zapotřebí všech těchto částí.
Zdrojový kód zapsaný v~programovacím jazyce Java je uložený v~souboru s~příponou \texttt{.java} (dále \texttt{java} souboru). Tento kód je při kompilaci převeden na mezikód, tzv. bajtkód, a uložen v~souborech s~příponou \texttt{.class} (dále \texttt{class} souborech). Bajtkód lze spustit pomocí virtuálního stroje, který má přístup k~Java API. O~tomto stroji lze  hovořit z~hlediska abstraktní specifikace nebo konkrétní implementace. Konkrétní implementace je závislá na daném systému a hardwaru, ale jednotná interpretace \texttt{class} souborů napříč platformami je zajištěna dodržením specifikace a platformově nezávislým formátem \texttt{class} souborů. Vzhledem k~zaměření práce se v~této kapitole zabývám abstraktní specifikací JVM dle dokumentu \cite{Lindholm:JVM}. 

\section{Datové typy a hodnoty}\label{JVMTypes}

Datové typy a hodnoty podporované JVM jsou znázorněné na obrázku \ref{figTypes}. Celá čísla jsou reprezentovaná datovými typy \texttt{byte} o~délce 8 bitů, \texttt{short} o~délce 16 bitů, \texttt{int} o~délce 32 bitů nebo \texttt{long} o~délce 64 bitů. Pro čísla s~plovoucí čárkou jsou definovány typy \texttt{float} o~32 bitech a \texttt{double} o~64 bitech. Typ \texttt{boolean} reprezentuje pravdivostní hodnoty pravda a nepravda. Typ \texttt{returnAddress} reprezentuje ukazatel na instrukci v~instrukčním souboru.

Znaky a řetězce jsou v~Javě kódované podle standardu Unicode v~kódování UTF-16, kde jeden znak je kódovaný jednou nebo dvěma kódovými jednotkami. Kódovou jednotku reprezentuje typ \texttt{char}. Jedná se o~16-bitové nezáporné číslo. Řetězec je pak reprezentovaný pomocí pole hodnot typu \texttt{char}. V~instrukčním souboru jsou řetězcové konstanty kódované v~modifikovaném kódování UTF-8. 

Typ \texttt{reference} označuje referenční datový typ a reprezentuje referenci na dynamicky vytvořený objekt. Podle toho, zda objekt je instancí třídy, pole nebo instancí třídy či pole, které implementují nějaké rozhraní, se rozlišuje typ reference. Hodnotou typu \texttt{reference} může být též speciální hodnota \texttt{null}, tedy reference na žádný objekt. 


% TODO vylepšit, ošklivé

\begin{figure}[!h]
\centering

\begin{tikzpicture}[level distance=1.6in,sibling distance=.1in,scale=.6]
\tikzset{
  %edge from parent/.style= {thick, draw},
  every tree node/.style={draw,minimum width=.7in,text width=1in, align=center}, 
  edge from parent fork right,
  grow'=right,
}

\Tree
[.{datové typy}
[.{primitivní datové typy} 
  [.{numerické datové typy} 
    [.{celočíselné}
      [{\texttt{byte}} ]
      [{\texttt{short}} ]
      [{\texttt{int}} ]
      [{\texttt{long}} ]
      [{\texttt{char}} ]
    ] 
    [.{s plovoucí čárkou}
      [{\texttt{float}} ]
      [{\texttt{double}} ]
    ]
  ] 
  [.{logický datový typ}
      [{\texttt{boolean}} ]
  ] 
  [{\texttt{returnAddress}}]
]]
[.{referenční datový typ} [{\texttt{reference}} ]]
]

\end{tikzpicture}

\caption{Datové typy a hodnoty podporované JVM.}\label{figTypes}
\end{figure}



Instrukční sada JVM je omezená a nenabízí u~všech instrukcí podporu pro všechny datové typy. Celá čísla jsou primárně reprezentovaná datovými typy \texttt{int} a \texttt{long} a případně přetypovaná na jeden z~typů \texttt{byte}, \texttt{short} či \texttt{char}. Pro datový typ \texttt{boolean} existuje jen podpora pro přetypování a pro vytvoření pole hodnot typu \texttt{boolean}. Pro výpočet logických výrazů se používá typ \texttt{int} s~hodnotami 0 a 1.


\section{Paměťové oblasti}\label{JVMData}

JVM pracuje s~několika typy paměťových oblastí. Velikosti těchto oblastí mohou být pevně dané, nebo se mohou měnit dynamicky podle potřeby. Při spuštění JVM vzniká halda a oblast metod. Halda je paměť určená pro alokaci instancí tříd a polí. Alokovanou paměť nelze dealokovat explicitně. Haldu automaticky spravuje tzv. \textit{garbage collector}. Oblast metod slouží k~ukládání kompilovaného kódu. Pro každou načtenou třídu se do této paměti ukládají struktury definující tuto třídu. Jednou z~těchto struktur je tzv. \textit{run-time constant pool}. Jedná se o~tabulku konstant z~\texttt{class} souboru, kde jsou symbolické reference na třídy, metody a členské proměnné nahrazeny konkrétními referencemi. Více se o~tabulce konstant zmiňuje kapitola \ref{FormatConstants}. 

Každá aplikace je spuštěna v~samostatném vlákně a při běhu mohou vznikat a zanikat i další vlákna. Všechna taková vlákna sdílí přístup k~haldě a oblasti metod. Navíc má každé vlákno k~dispozici vlastní \texttt{pc} registr a zásobník rámců. V~\texttt{pc} registru je uchováván ukazatel na aktuálně vykonávanou instrukci, není-li aktuálně vykonávaná metoda nativní. Zásobník rámců obsahuje data zavolaných metod. Při každém volání metody je vytvořen nový rámec. Rámec má vlastní pole lokálních proměnných a operační zásobník. V~poli lokálních proměnných se uchovávají hodnoty parametrů a lokálních proměnných. Hodnoty lze vkládat na operační zásobník, provádět nad nimi výpočty a ukládat zpět do pole. Operační zásobník slouží k~předávání operandů instrukcím a k~uchovávání mezivýsledků. Podílí se také na předávání parametrů a návratových hodnot.

Před voláním metody je třeba nejprve vložit parametry na operační zásobník aktuálního rámce. Zavoláním metody se vytvoří nový rámec a umístí se na vrchol zásobníku rámců. Parametry se následně přesunou z~operačního zásobníku předchozího rámce do pole lokálních proměnných nového rámce. Registr \texttt{pc} se nastaví na první instrukci volané metody a začnou se vykonávat jednotlivé instrukce. Při návratu z~metody je návratová hodnota umístěná na vrcholu operačního zásobníku, vrací-li metoda nějakou hodnotu. Tato hodnota je přesunuta na operační zásobník předcházejícího rámce a aktuální rámec je ze zásobníku rámců odstraněn. Dojde k~obnovení stavu volající metody. Registr \texttt{pc} je nastaven na index instrukce, která bezprostředně následuje za instrukcí volající metodu. Pokud je metoda ukončena vyvoláním nezachycené výjimky, pak k~předání hodnoty nedochází. Aktuální rámec je odstraněn a výjimka je znovu vyvolaná ve volající metodě. Zpracování výjimek je více vysvětleno v~kapitole \ref{FormatMethod}.


Nejmenší jednotka, se kterou pracuje operační zásobník, je 32-bitová hodnota. Hodnoty většiny datových typů lze vyjádřit pomocí jedné jednotky, ale hodnoty typů \texttt{long} a \texttt{double} je třeba reprezentovat dvěma jednotkami. S~takovou dvojicí je třeba vždy manipulovat jako s~celkem. Datový typ hodnoty na zásobníku je daný instrukcí, která ho tam vložila, a na hodnotu nelze nahlížet jinak. Hloubka operačního zásobníku je určená počtem jednotek na zásobníku. Proto se o~jednotce dále zmiňuji jako o~jednotce hloubky zásobníku.

Lokální proměnná v~poli lokálních proměnných je 32-bitová hodnota. Proměnnou lze adresovat pomocí indexu do pole, kde pole je indexováno od nuly. Lokální proměnná může být typu \texttt{byte}, \texttt{short}, \texttt{int}, \texttt{char}, \texttt{float}, \texttt{boolean}, \texttt{reference} nebo \texttt{returnAddress}. Hodnoty typu \texttt{long} a \texttt{double} jsou uchovávané pomocí dvojice lokálních proměnných. V~tom případě k~adresaci slouží nižší z~indexů a na větší index se nesmí přistupovat. 


\section{Kontrola instrukčního souboru}\label{JVMVerification}

Při načítání \texttt{class} souboru je ověřeno správné formátování tak, jak je popsáno v~kapitole \ref{Format}. Jsou zkontrolovány první čtyři bajty souboru, předdefinované atributy musí být správné délky, názvy a typy tříd, rozhraní, metod a proměnných musí být validní, indexy do tabulky konstant musí adresovat správný typ položky. Dále jsou kladena jistá omezení na kód metod. Mimo jiné, argumenty instrukcí musí mít správný typ a musí být správného počtu, integrita hodnot typu \texttt{long} a \texttt{double} nemůže být nikdy narušena, k~hodnotě lokální proměnné se nesmí přistupovat před inicializací proměnné, nesmí dojít k~načtení hodnoty z~prázdného zásobníku a všechny metody musí byt ukončené \texttt{return} instrukcí. Verifikace těchto omezení se provádí pomocí typové kontroly nebo typové inference. Kontrola těchto omezení se tak nemusí provádět za běhu programu.


%=========================================================================
