%%%%%%%%%%%%%%%%%%%%%%%%%%%%%%%%%%%%%%%%%%%%%%%%%%%%%%%%%%%%%%%%%%%%%%%%%%
\chapter{Závěr}\label{Conclusion}

% Shrnutí toho, co jsem udělala, jakých výsledků jsem dosáhla a jak budu pokračovat.

Na základě specifikace \cite{Lindholm:JVM} jsem popsala virtuální stroj Java Virtual Machine a formát jeho instrukčního souboru.  Dále jsem se seznámila s~nástroji pro manipulaci s~bajtkódem BCEL, ASM a Javassist, uvedla jejich stručný popis a vzájemně je porovnala. Tyto knihovny jsem následně použila pro implementaci nástroje \texttt{jbyca} pro analýzu bajtkódu. Nástroj mi umožnil analyzovat velký vzorek \texttt{class} souborů a získat data vhodná k návrhu metod pro optimalizaci velikosti souborů. Konkrétně jsem se zabývala počty a velikostmi položek v souborech, využitím paměťového prostoru pro lokální proměnné a analýzou typických sekvencí instrukcí. Na základě poznatků z této analýzy jsem navrhla metody pro optimalizace velikosti bajtkódu. Většina navržených optimalizací je založena na náhradě sekvence instrukcí za kratší sekvenci. Další slouží k odstranění neužitečných informací ze souborů, a nebo vedou k modifikaci struktury programu. Některé z navržených optimalizačních metod jsem následně implementovala v nástroji \texttt{jbyco} pro optimalizaci velikosti bajtkódu. Při návrhu nástroje jsem kladla důraz na snadnou modifikovatelnost a rozšiřovatelnost implementovaných optimalizací. Nakonec jsem s pomocí tohoto nástroje optimalizovala testovací vzorek dat a výstupy opět analyzovala. Velikost dat se snížila o 22\% se zachováním původních struktur programů.

Jako možné pokračování této práce se nabízí optimalizace nástroje pro hledání a nahrazování sekvencí instrukcí, jak je popsáno v kapitole \ref{}.
Dále by bylo vhodné implementovat a analyzovat ostatní navrhnuté metody včetně těch, které modifikují strukturu programu. Je také možné rozšířit oblast aplikace optimalizačních metod z lokální úrovně na úroveň interprocedurální a intraprocedurální. Optimalizace na takové úrovni však již vyžadují vhodnější reprezentaci instrukcí a pokročilé analýzy kódu.


%=========================================================================
