%%%%%%%%%%%%%%%%%%%%%%%%%%%%%%%%%%%%%%%%%%%%%%%%%%%%%%%%%%%%%%%%%%%%%%%%%%
\chapter{Závěr}\label{Conclusion}

% Shrnutí toho, co jsem udělala, jakých výsledků jsem dosáhla a jak budu pokračovat.

% Popsala jsem virtuální stroj Java Virtual Machine a formát jeho instrukčního souboru. Vycházela jsem ze specifikace \cite{Lindholm:JVM} s~cílem podat ji trochu jiným způsobem. Seznámila jsem se s~nástroji pro manipulaci s~bajtkódem BCEL, ASM, a Javassist, uvedla jejich stručný popis a vzájemně je porovnala. Dále jsem navrhla a implementovala vlastní nástroj \texttt{jbyco} pro analýzu bajtkódu. Stáhla jsem si velké množství \texttt{jar} souborů a pomocí \texttt{jbyco} jsem získala data, která jsem dále analyzovala. Při analýze jsem se zabývala celkovou velikostí položek v~instrukčním souboru, sekvencemi instrukcí s~častým výskytem a využitím lokálních proměnných.

TODO

%=========================================================================
